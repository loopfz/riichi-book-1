%~~~~~~~~~~~~~~~~~~~~~~~~~~~~~~~~~~~~~~~~~~~~~~~~~
% riichi Book 1, Chapter 4: Five block method
%~~~~~~~~~~~~~~~~~~~~~~~~~~~~~~~~~~~~~~~~~~~~~~~~~

\chapter{La méthode des cinq blocs} \label{ch:five}
\thispagestyle{empty}

En présentant les éléments de base du riichi dans le chapitre précédent, j'ai également abordé un certain nombre de principes importants en matière d'efficacité des tuiles ---.
Par exemple, la supériorité des proto-suites en attente sur les côtés, l'intérêt d'avoir deux paires dans une main plutôt que trois, et l'intérêt d'étirer les formes de flotteurs simples ou bombés, pour n'en citer que quelques-uns. 

\bigskip
Ces principes sont tous importants, mais essayer de prendre en compte tous les principes importants en même temps peut être une tâche décourageante. Nous devons choisir notre défausse en un temps limité,\footnote{Rappelons que sur les tables normales (plus lentes) du {\jap Tenhou}, chaque choix de défausse doit être fait en moins de 5 secondes. Dans les parties hors ligne, nous devrions faire des choix encore plus rapides pour ne pas irriter les autres joueurs, et l'efficacité des tuiles n'est pas le seul facteur à prendre en compte pour faire un choix de défausse. 
De plus, certains principes d'efficacité des tuiles peuvent parfois entrer en conflit les uns avec les autres, ce qui nous oblige à choisir le principe à suivre. Par exemple, nous pouvons parfois nous demander s'il faut conserver une forme de flotteur bombé ou conserver deux paires dans une main, lorsque nous devons défausser l'une des deux. 

\bigskip
La méthode des cinq blocs que je présente dans ce chapitre nous aidera à établir des priorités entre des principes concurrents et à trouver rapidement le choix de défausse le plus efficace.\N- Comme je l'ai mentionné dans la préface, l'exposé de ce chapitre est basé sur les livres de Makoto Fukuchi. En particulier, je suis redevable à Makoto Fukuchi 2015 \textit{Haikouritsu Nyumon Doriru 76}, Yousensha. ISBN978-4-8003-0634-0.}
L'idée de base de la méthode des cinq blocs est d'une simplicité trompeuse ; nous devons d'abord indentifier cinq blocs de tuiles dans une main --- quatre groupes + une paire, ou leurs candidats --- et essayer de compléter chaque bloc. 
\index{Fukuchi@Fukuchi, Makoto}

\section{Recherche d'une tuile redondante} 
	\index{méthode des cinq blocs}

Nous savons tous qu'une main standard doit avoir cinq blocs de tuiles --- quatre groupes et une paire --- pour gagner.
%\footnote{Bien sûr, les mains non standard --- {\jap chiitoitsu} (Sept paires) et {\jap kokushimusou} (Treize orphelins) --- sont des exceptions.} 
La méthode des cinq blocs nous encourage à être toujours conscients des cinq blocs de tuiles dans une main. 
Considérez la main suivante. Que rejetteriez-vous et pourquoi ?

\bp
\wan{5}\wan{5}\wan{7}\wan{8}\tong{2}\tong{4}\tong{5}\tong{5}\suo{1}\suo{3}\suo{3}\zhong\zhong\zhong
\ep

Pour déterminer quelle tuile est la moins utile dans cette main, divisons la main en blocs de tuiles, comme suit. 
\bmj{\Huge
$
\underbrace{\text{\wan{5}\wan{5}}}
\underbrace{\text{\wan{7}\wan{8}}}\underbrace{\text{\tong{2}\tong{4}\tong{5}\tong{5}}}\underbrace{\text{\suo{1}\suo{3}\suo{3}}}\underbrace{\text{\zhong\zhong\zhong}} \label{five:411a}
$
}\emj
Remarquez que, bien que nous ne sachions pas encore quel bloc sera la paire et quels blocs seront les quatre groupes, la main possède déjà cinq blocs de tuiles. Cela signifie qu'il n'est pas nécessaire d'augmenter ou de diminuer le nombre de blocs à partir de maintenant. 

\bigskip
En regardant chacun des cinq blocs, la paire de {\LARGE\wan{5}}, le protorun {\LARGE\wan{7}\wan{8}}, et l'ensemble de {\LARGE\zhong} sont tous auto-suffisants ; nous les gardons tels quels. Notre choix de défausse doit donc se porter sur le troisième ou le quatrième bloc, {\LARGE\tong{2}\tong{4}\tong{5}\tong{5}} ou {\LARGE\suo{1}\suo{3}\suo{3}}. 
Comparons maintenant ces deux blocs d'attente fermée. Alors que {\LARGE\suo{1}} est utile dans le bloc auquel il appartient, permettant à la main d'accepter {\LARGE\suo{2}}, {\LARGE\tong{2}} est complètement redondant ; la main peut accepter {\LARGE\tong{3}} sans avoir {\LARGE\tong{2}}. La défausse idéale est donc {\LARGE\tong{2}}. 

\bigskip
Il y a deux points essentiels à retenir dans l'application de la méthode des cinq blocs. 
Tout d'abord, aucun des cinq blocs ne doit être "trop faible". \footnote{Basiquement, tout bloc qui est plus faible qu'un protorun d'attente latérale est un bloc faible.} Dans l'exemple actuel, si nous écartons {\LARGE\suo{3}}, le bloc {\LARGE\suo{1}\suo{3}\suo{3}} devient un protorun fermé-attendu isolé, qui est trop faible par rapport aux autres blocs. De même, si l'on écarte {\LARGE\suo{1}}, ce bloc devient une paire de {\LARGE\suo{3}}. Comme cette main possède déjà deux autres paires, la présence d'une troisième paire rend la main trop faible. 

\bigskip
Deuxièmement, chacun des blocs de cinq tuiles devrait idéalement comporter trois tuiles. Dans l'exemple actuel, le bloc {\LARGE\suo{1}\suo{3}\suo{3}} a exactement trois tuiles et nous ne devrions donc pas choisir une défausse dans ce bloc. En revanche, le bloc {\LARGE\tong{2}\tong{4}\tong{5}\tong{5}} comporte actuellement quatre tuiles et nous devons donc en écarter une pour en faire un bloc à trois tuiles. 

\vfill
\color{MyRed}
\begin{itembox}[c]{Méthode des cinq blocs}\normalcolor
Identifier les blocs à cinq tuiles dans une main. Essayez de vous assurer : 
\bi\itemsep.1pt
\i[] (1) qu'aucun bloc n'est trop faible ; et 
\i[] (2) que chaque bloc a au plus trois tuiles. 
\ei
\end{itembox}\normalcolor

\newpage
Prenons un autre exemple. 
\bp
\wan{3}\wan{5}\wan{7}\tong{4}\tong{5}\tong{6}\tong{6}\tong{7}\suo{4}\suo{6}\suo{6}\suo{8}\bai\bai~\bai\\
\hfill\footnotesize{{\jap Dora}~~~~~~~~~~~~}
\ep
Nous pouvons facilement voir qu'il y a un bloc dans {\jap manzu} (fissures), deux blocs dans {\jap pinzu} (points), et une paire de dragons blancs, ce qui nous donne quatre blocs. Cela signifie que nous n'avons besoin que d'un seul bloc supplémentaire dans {\jap souzu} (bambous). Nous divisons donc la main comme suit.
\bmj{\Huge
$
\underbrace{\text{\wan{3}\wan{5}\wan{7}}}
\underbrace{\text{\tong{4}\tong{5}\tong{6}}} \underbrace{\text{\tong{6}\tong{7}}}\underbrace{\text{\suo{4}\suo{6}\suo{6}\suo{8}}}\underbrace{\text{\bai\bai}} \label{five:411b}
$
}\emj
Puisque nous ne devons pas créer un bloc trop faible, écarter {\LARGE\wan{3}} ou {\LARGE\wan{7}} n'est pas une option. Remarquez que le bloc de {\jap souzu} (bambous) a quatre tuiles. Nous devons donc en écarter une de ce bloc. Au cas où la paire de dragons blancs deviendrait plus tard un brelan, nous devrions conserver la paire de {\LARGE\suo{6}}, laissant {\LARGE\suo{4}} ou {\LARGE\suo{8}} comme candidat à la défausse. Étant donné que {\LARGE\suo{4}} a plus de chances de créer un protorun d'attente latérale, nous devrions écarter {\LARGE\suo{8}}. 
Ainsi, aucun des cinq blocs n'est trop faible, et chaque bloc a au plus trois tuiles. 

\bigskip
Dans les deux exemples que nous avons vus ci-dessus, vous avez peut-être pu identifier les tuiles redondantes sans trop réfléchir. Si c'est le cas, c'est probablement parce que vous avez appliqué implicitement et inconsciemment la méthode des cinq blocs dans votre esprit. L'objectif de ce chapitre est d'entraîner davantage notre esprit, de sorte que l'identification des blocs de cinq tuiles dans une main devienne une seconde nature. 

\section{Configurations alternatives}
Considérez la main suivante. Que rejetteriez-vous et pourquoi ?

\bp
\wan{3}\wan{3}\wan{4}\wan{6}\tong{2}\tong{2}\tong{4}\tong{5}\tong{6}\tong{6}\tong{7}\zhong\zhong\zhong
\ep

Divisons d'abord la main en cinq blocs de tuiles. 

\bmj{\cHuge
$
\underbrace{\text{\wan{3}\wan{3}\wan{4}\wan{6}}}
\underbrace{\text{\tong{2}\tong{2}}}\underbrace{\text{\tong{4}\tong{5}\tong{6}}}\underbrace{\text{\tong{6}\tong{7}}}\underbrace{\text{\zhong\zhong\zhong}} \label{five:421a}
$
}\emj

Il apparaît ainsi plus clairement que, comme dans l'exemple précédent, {\LARGE\wan{6}} crée un protorun d'attente fermé redondant, et qu'il faut donc l'écarter. De plus, l'élimination de {\LARGE\wan{6}} fait de ce bloc un bloc à trois tuiles. 

\N- Bigskip
Cependant, il existe une autre façon de diviser cette main en cinq blocs, et les changements de situation peuvent nécessiter une telle alternative. 
Supposons que vos adversaires aient déjà défaussé les quatre tuiles de {\LARGE\wan{2}}. Supposons également que {\LARGE\tong{3}} semble présent dans le mur. Ou supposons que les tuiles {\LARGE\wan{3}-\wan{6}} semblent trop dangereuses pour être défaussées contre un adversaire. Dans ce cas, nous pourrions vouloir diviser la main de la manière suivante. 

\bmj{\Huge
$
\underbrace{\text{\wan{3}\wan{3}}}
\underbrace{\text{\wan{4}\wan{6}}}\underbrace{\text{\tong{2}\tong{2}\tong{4}\tong{5}\tong{6}\tong{6}\tong{7}}}_{\text{\small 2}}\underbrace{\text{\zhong\zhong\zhong}} \label{five:421b}
$
}\emj

En d'autres termes, notre objectif est de faire de la paire de {\LARGE\wan{3}} la paire de cette main, et nous cherchons à avoir deux séries de {\jap pinzu} (points). Si nous écartons {\LARGE\tong{2}}, ce bloc devient {\LARGE\tong{2}\tong{4}\tong{5}\tong{6}\tong{6}\tong{7}}. 
Rappelons qu'un tel bloc peut être divisé en {\LARGE\tong{2}\tong{4}\tong{6} + \tong{5}\tong{6}\tong{7}} (rappelons la discussion sur la double forme fermée dans la section \ref{sec:ryankan}). Par conséquent, ce bloc peut accepter {\LARGE\tong{3}} ainsi que {\LARGE\tong{5}-\tong{8}} pour réaliser deux suites dans {\jap pinzu} (points). Le bloc de {\jap pinzu} (points) aura six tuiles, mais ce n'est pas grave car ce bloc en vaut deux. 

\N- Bigskip
Pour maîtriser la méthode des cinq blocs, nous devons être capables d'envisager instantanément la configuration du premier bloc (\ref{five:421a}) dès que nous voyons cette main. Mais cela ne suffit pas. Nous devons également être capables d'imaginer une configuration alternative (\ref{five:421b}) au même moment. 
Au mahjong, les situations changent très rapidement à chaque fois qu'une nouvelle tuile est piochée ou qu'une nouvelle tuile est défaussée. Par conséquent, la configuration idéale des cinq blocs changera également en fonction de l'évolution de la situation. Nous devons donc développer nos capacités à imaginer de nombreuses configurations possibles de cinq blocs et à nous préparer à d'éventuels changements de situation qui nécessiteraient une modification de la configuration.

\Grandes lignes
Je propose plusieurs exercices dans les pages suivantes. Le corrigé de chaque exercice se trouve à la page suivante. Essayez de ne pas regarder les réponses avant d'avoir trouvé votre propre réponse. 

\vfill

\subsection*{Exercices : trouver une tuile redondante}

\bigskip

\begin{itembox}[l]{Exercice 1}
Que jetteriez-vous ? 
\vsp
Comment diviser la main en cinq blocs de tuiles ?

\bp
\wan{1}\wan{1}\wan{3}\wan{4}\tong{3}\tong{4}\tong{5}\suo{3}\suo{3}\suo{5}\suo{6}\suo{7}\suo{7}~\suo{4}\\
\hfill\footnotesize{Draw~~~~~~~~~~~}
\ep
\end{itembox}
%=====================

\newpage

\begin{itembox}[r]{Réponse 1}
\bmj{\Huge
$
\underbrace{\text{\wan{1}\wan{1}}}%
\underbrace{\text{\wan{3}\wan{4}}}~\underbrace{\text{\tong{3}\tong{4}\tong{5}}}~\underbrace{\text{\suo{3}\suo{3}\suo{4}\suo{5}\suo{6}\suo{7}\suo{7}}}_{\text{\small 2}} \nonumber
$
}\emj
With the draw of {\LARGE\suo{4}}, we now have a 3-way side-wait block in {\jap souzu} (bamboos). {\LARGE\suo{3}} or {\LARGE\suo{7}} could be our back-up candidate for the head, in case we draw another {\LARGE\wan{1}}. Since there is {\jap sanshoku} (Mixed Triple Chow) of 345, we discard {\LARGE\suo{3}}. 
\end{itembox}

\vfill

\begin{itembox}[l]{Exercise 2}
What would you discard? \\
\vsp
How do you divide the hand into five tile blocks? 

\bp
\wan{5}\wan{7}\tong{1}\tong{1}\tong{2}\tong{3}\tong{4}\tong{4}\tong{5}\tong{7}\suo{3}\suo{5}\suo{5}~\tong{4}\\
\hfill\footnotesize{Draw~~~~~~~~~~~}
\ep
\end{itembox}
%=====================

\newpage


%=====================
\bigskip
\begin{itembox}[r]{Answer 2}
\bmj{\Huge
$
\underbrace{\text{\wan{5}\wan{7}}}\text{\tong{1}}
\underbrace{\text{\tong{1}\tong{2}\tong{3}}}\underbrace{\text{\tong{4}\tong{4}\tong{4}}}\underbrace{\text{\tong{5}\tong{7}}}\underbrace{\text{\suo{3}\suo{5}\suo{5}}}
\nonumber
$
}\emj
Before we drew the third {\LARGE\tong{4}}, the {\jap pinzu} (dots) tiles were {\LARGE\tong{1}\tong{1} + \tong{2}\tong{3}\tong{4} + \tong{4}\tong{5}}, so the {\LARGE\tong{7}} was simply a floating tile. Now that we have another {\LARGE\tong{4}}, the five-block configuration changes accordingly. The ideal discard is {\LARGE\tong{1}}, as this has become redundant.
\end{itembox}
%=====================

\vfill

\begin{itembox}[l]{Exercise 3}
What would you discard? \\
\vsp
How do you divide the hand into five tile blocks? 

\bp
\wan{4}\wan{4}\wan{6}\wan{6}\wan{7}\wan{8}\tong{2}\tong{3}\suo{3}\suo{4}\suo{6}\suo{6}\suo{7}~\wan{3}\\
\hfill\footnotesize{Draw~~~~~~~~~~~}
\ep
\end{itembox}
%=====================


\newpage

\begin{itembox}[r]{Answer 3}
\bmj{\Huge
$
\underbrace{\text{\wan{3}\wan{4}\wan{4}}}\text{\wan{6}}
\underbrace{\text{\wan{6}\wan{7}\wan{8}}}\underbrace{\text{\tong{2}\tong{3}}}\underbrace{\text{\suo{3}\suo{4}}}\underbrace{\text{\suo{6}\suo{6}\suo{7}}}
\nonumber
$
}\emj
There are two ``side wait plus one'' shapes, {\LARGE\wan{3}\wan{4}\wan{4}} and {\LARGE\suo{6}\suo{6}\suo{7}}, that might later become the head or a run. At this point, however, we cannot determine which one will be which, so we should keep them as they are. 
One of the two {\LARGE\wan{6}} has become an obvious redundancy so we should discard one. 
\end{itembox}

\vfill

\begin{itembox}[l]{Exercise 4}
What would you discard? \\
\vsp
How do you divide the hand into five tile blocks? 

\bp
\wan{2}\wan{3}\wan{5}\wan{6}\tong{4}\tong{4}\tong{6}\tong{8}\tong{9}\suo{3}\suo{4}\suo{5}\bei~\wan{4}\\
\hfill\footnotesize{Draw~~~~~~~~~~~}
\ep
\end{itembox}
%=====================


\newpage

\begin{itembox}[r]{Answer 4}
\bmj{\Huge
$ 
\underbrace{\text{\wan{2}\wan{3}\wan{4}\wan{5}\wan{6}}}_{\text{\small 2}}
\underbrace{\text{\tong{4}\tong{4}}}\underbrace{\text{\tong{6}\tong{8}}}\text{\tong{9}}\underbrace{\text{\suo{3}\suo{4}\suo{5}}}\text{\bei}
\nonumber
$ 
}\emj
The {\LARGE\bei} is obviously redundant, but {\LARGE\tong{9}} is also useless. Without {\LARGE\tong{9}}, the hand can accept {\LARGE\tong{7}}. Since honor tiles can be used as a safety tile (see Chapter \ref{ch:defense}), we discard {\LARGE\tong{9}} first. 
\end{itembox}
%=====================

\vfill


\begin{itembox}[l]{Exercise 5}
What would you discard? \\
\vsp
How do you divide the hand into five tile blocks? 

\bp
\wan{2}\wan{3}\wan{5}\wan{7}\wan{7}\tong{2}\tong{4}\tong{6}\tong{8}\suo{4}\suo{5}\suo{7}\suo{8}~\suo{8}~\tong{4}\\
\hfill\footnotesize{Draw~~{\jap Dora}~~~~~~~}
\ep
\end{itembox}
%=====================

\newpage

\begin{itembox}[r]{Answer 5}
\bmj{\Huge
$ 
\underbrace{\text{\wan{2}\wan{3}}}
\underbrace{\text{\wan{5}\wan{7}\wan{7}}}\underbrace{\text{\tong{2}\tong{4}\tong{6}\tong{8}}}\underbrace{\text{\suo{4}\suo{5}}}\underbrace{\text{\suo{7}\suo{8}\suo{8}}}
\nonumber
$
}\emj
This is a bit difficult, as there are so many closed-wait protoruns. Recall that each tile block should have at most three tiles and that pairs are most valuable when there are two of them in a hand. 
The block in {\jap pinzu} (dots) has four tiles, so we discard one from this block. Since {\LARGE\tong{4}} is {\jap dora}, we discard {\LARGE\tong{8}}, leaving the double closed shape around {\jap dora}: {\LARGE\tong{2}\tong{4}\tong{6}}. 
\end{itembox}
%=====================

\vfill

\begin{itembox}[l]{Exercise 6}
What would you discard? \\
\vsp
How do you divide the hand into five tile blocks? 

\bp
\wan{2}\wan{3}\wan{4}\wan{4}\tong{3}\tong{4}\tong{5}\tong{6}\suo{4}\suo{5}\suo{5}\suo{6}\suo{6}~\tong{4}\\
\hfill\footnotesize{Draw~~~~~~~~~~~}
\ep
\end{itembox}
%=====================

\newpage

\begin{itembox}[r]{Answer 6}
\bmj{\Huge
$ 
\underbrace{\text{\wan{2}\wan{3}\wan{4}}}\text{\wan{4}}
\underbrace{\text{\tong{3}\tong{4}\tong{4}\tong{5}\tong{6}}}_{\text{\small 2}}
\underbrace{\text{\suo{4}\suo{5}\suo{5}\suo{6}\suo{6}}}_{\text{\small 2}}
\nonumber
$
}\emj
Finding the best discard by actually comparing tile acceptance counts for each possible discard candidate is super tedious. The five-block method simplifies the process quite a bit. Since we have two blocks in {\jap pinzu} (dots) and two blocks in {\jap souzu} (bamboos), we only need one block in {\jap manzu} (cracks), hence one {\LARGE\wan{4}} is redundant. If we discard {\LARGE\wan{4}}, the hand can be made ready with 11 kinds--34 tiles. If we discard {\LARGE\tong{3} \tong{4}} or {\LARGE\suo{5}}, the hand can be ready only with 6 kinds--19 tiles. 
\end{itembox}
%=====================

\vfill

\begin{itembox}[l]{Exercise 7}
What would you discard? 
\vspace{-10pt}
%How do you divide the hand into five tile blocks? 

\bp
\wan{3}\wan{4}\wan{7}\tong{2}\tong{3}\tong{4}\tong{7}\tong{7}\tong{8}\suo{4}\suo{5}\suo{7}\suo{9}~\suo{6}\\
\hfill\footnotesize{Draw~~~~~~~~~~~}
\ep
\end{itembox}
%=====================

\newpage

%=====================

\begin{itembox}[r]{Answer 7}
\bmj{\Huge
$
\underbrace{\text{\wan{3}\wan{4}}}\text{\wan{7}}
\underbrace{\text{\tong{2}\tong{3}\tong{4}}}\underbrace{\text{\tong{7}\tong{7}\tong{8}}}\underbrace{\text{\suo{4}\suo{5}\suo{6}\suo{7}\suo{9}}}_{\text{\small 2}}\nonumber
$
}\emj
Do not discard the {\LARGE\suo{9}} just because it forms a closed wait or because discarding it gets us {\jap tanyao} (All Simples). Avoiding closed wait too much and being hung up on {\jap tanyao} are two pathologies common among intermediate players. 
\bigskip

The block in {\jap souzu} (bamboos) is actually not too bad; this is a stretched single plus one, which can become either two runs immediately (if we draw {\LARGE\suo{8}}), one run plus one side-wait protorun (if we draw any of {\LARGE\suo{3}\suo{5}\suo{6}}), or one run plus the head (if we draw {\LARGE\suo{4}} or {\LARGE\suo{7}}). Note also that we need both {\LARGE\tong{7}} and {\LARGE\tong{8}}, because this part may become the head if we get two runs in {\jap souzu} (bamboos); when we get the head in {\jap souzu} (bamboos), we will treat this part as a side-wait protorun. We thus discard {\LARGE\wan{7}}. 
\end{itembox}
%=====================


\newpage

\section{Sélection des blocs de tuiles}
All the hands we have seen so far in this chapter already have five tile blocks. In practice, however, this is not always the case. A hand can sometimes have fewer or more tile blocks. Since we need to have exactly five blocks to win a hand, we will need to bump up tile blocks by using a floating tile when we have fewer of them or to discard some blocks entirely when we have a plethora of them. 

\bigskip
In selecting which tile blocks to keep and which ones to discard, we focus on a combination of the following three criteria: 
\be
\i tile efficiency;
\i hand value;
\i the safety of tiles to be discarded.
\ee
As we will see below, we can sometimes find a block to discard based on all the three criteria. 
Consider the following hand. How do we divide the hand into tile blocks, and what would you discard?
%\vspace{-20pt}

\bp 
\hspace{120pt}{\footnotesize\color{red!75!black} Red}\\ \vspace{-16pt}
\wan{1}\wan{2}\wan{3}\tong{3}\tong{4}\tong{5}\tong{5}\tong{7}\tong{8}\suo{4}\rfs\suo{7}\suo{8}~\tong{2}\\
\hfill\footnotesize{Draw~~~~~~~~~~~~~~~}
\ep
We can see that the hand currently has six tile blocks, as follows. 
\bmj{\Huge
$ 
\underbrace{\text{\wan{1}\wan{2}\wan{3}}}
\underbrace{\text{\tong{2}\tong{3}\tong{4}}}
\underbrace{\text{\tong{5}\tong{5}}}
\underbrace{\text{\tong{7}\tong{8}}}
\underbrace{\text{\suo{4}\rfs}}
\underbrace{\text{\suo{7}\suo{8}}}
\nonumber
$
}\emj
Since the first two tile blocks are already complete and the third block is the head, our discard choice should be from the last three tile blocks, {\LARGE\tong{7}\tong{8}}, {\LARGE\suo{4}\rfs}, or {\LARGE\suo{7}\suo{8}}. 

\bigskip
From a perspective of tile efficiency, discarding the {\LARGE\tong{7}\tong{8}} block means that we lose the ability to accept \emph{both} {\LARGE\tong{6}} and {\LARGE\tong{9}}. On the other hand, if we discard the {\LARGE\suo{7}\suo{8}} block, we only lose the ability to accept {\LARGE\suo{9}}; because of the {\LARGE\suo{4}\rfs} block, we can still accept {\LARGE\suo{6}}. We should thus choose between the two blocks in {\jap souzu} (bamboos). 
Keeping the {\LARGE\rfs} is desirable from a perspective of hand value (it is a red five) as well as safety (discarding {\LARGE\suo{7}\suo{8}} is much safer than discarding {\LARGE\suo{4}\rfs}, generally speaking). Therefore, the three criteria collectively suggest that we should discard {\LARGE\suo{7}\suo{8}}. 

\bigskip

In practice, however, satisfying all three criteria may not be feasible. A common tradeoff we face is between speed and hand value. That is, maximizing tile efficiency to gain speed often entails giving up possibilities of pursuing an expensive hand. Consider the following hand. 
\bp
\wan{1}\wan{3}\wan{5}\wan{7}\wan{7}\tong{3}\tong{7}\tong{8}\suo{3}\suo{4}\suo{8}\suo{8}\suo{9}\suo{9}
\ep
Let's divide the hand into tile blocks. There are several ways to do this. One way to do this is to split it into the following blocks. 
\bmj{\Huge
$ 
\underbrace{\text{\wan{1}\wan{3}\wan{5}}}
\underbrace{\text{\wan{7}\wan{7}}}
\text{\tong{3}}
\underbrace{\text{\tong{7}\tong{8}}}
\underbrace{\text{\suo{3}\suo{4}}}
\underbrace{\text{\suo{8}\suo{8}}}
\underbrace{\text{\suo{9}\suo{9}}}
\nonumber
$
}\emj
If we simply maximize tile efficiency, we discard {\LARGE\tong{3}}, as we already have six tile blocks and we won't need any more floating tile. 

\bigskip
However, as it stands, the hand has no {\jap yaku} and it is likely to be a very cheap riichi-only hand. Moreover, the hand has three pairs, which is not ideal as we saw in the previous chapter. 
Therefore, we might want to split the hand into the following five blocks. 
\bmj{\Huge
$ 
\underbrace{\text{\wan{1}\wan{3}\wan{5}\wan{7}\wan{7}}}
\underbrace{\text{\tong{3}}}
\underbrace{\text{\tong{7}\tong{8}}}
\underbrace{\text{\suo{3}\suo{4}}}
\underbrace{\text{\suo{8}\suo{8}}}
\text{\suo{9}\suo{9}}
\nonumber
$
}\emj
We count the floating {\LARGE\tong{3}} as an independent block, hoping it to grow into a run. We also treat the tiles in {\jap manzu} (cracks) as a single block, hoping to get at least one group or the head out of it. We thus discard one {\LARGE\suo{9}} now, then another {\LARGE\suo{9}} in the next turn. Depending on what tile gets drawn, our five-block configuration will be different. 

\bigskip
For example, suppose we draw {\LARGE\tong{4}} and then {\LARGE\suo{5}}. We will then have the following. 
\bmj{\Huge
$ 
\underbrace{\text{\wan{1}\wan{3}\wan{5}\wan{7}\wan{7}}}
\underbrace{\text{\tong{3}\tong{4}}}
\underbrace{\text{\tong{7}\tong{8}}}
\underbrace{\text{\suo{3}\suo{4}\suo{5}}}
\underbrace{\text{\suo{8}\suo{8}}}
\nonumber
$
}\emj
We will discard the {\LARGE\wan{1}} as a first step toward reducing the number of tiles in the {\jap manzu} (cracks) block to three. We can now see that this hand has a potential of getting {\jap sanshoku} of 345 as well as {\jap pinfu} and {\jap tanyao}.

\bigskip
On the other hand, if we draw {\LARGE\wan{6}} and then {\LARGE\wan{8}}, we can expect to have two groups in {\jap manzu} (cracks) so we will discard {\LARGE\tong{3}}.
\bmj{\Huge
$ 
\underbrace{\text{\wan{1}\wan{3}\wan{5}\wan{6}\wan{7}\wan{7}\wan{8}}}_{\text{\small 2}}
\text{\tong{3}}
\underbrace{\text{\tong{7}\tong{8}}}
\underbrace{\text{\suo{3}\suo{4}}}
\underbrace{\text{\suo{8}\suo{8}}}
\nonumber
$
}\emj
\vspace{-10pt}

In selecting tile blocks, we should try to achieve the best balance between speed and hand value. Don't fantasize too much about getting an expensive hand. At the same time, don't fixate too much about tile efficiency at the cost of hand value. This is of course easier said than done; it is quite difficult even for advanced players. 

\vfill

\subsection*{Exercises: selecting tile blocks}

\bigskip

\begin{itembox}[l]{Exercise 8}
What would you discard? \\
\vsp
How do you divide the hand into five tile blocks? 

\bp
\wan{2}\wan{3}\wan{3}\wan{7}\wan{8}\tong{5}\tong{6}\suo{1}\suo{1}\suo{2}\suo{4}\suo{9}\suo{9}~\suo{9}\\
\hfill\footnotesize{Draw~~~~~~~~~~~}
\ep
\end{itembox}
%=====================

\newpage

\begin{itembox}[r]{Answer 8}
\bmj{\Huge
$ 
\underbrace{\text{\wan{2}\wan{3}\wan{3}}}
\underbrace{\text{\wan{7}\wan{8}}}
\underbrace{\text{\tong{5}\tong{6}}}
\underbrace{\text{\suo{1}\suo{1}}}
\underbrace{\text{\suo{2}\suo{4}}}
\underbrace{\text{\suo{9}\suo{9}\suo{9}}}
\nonumber
$
}\emj
The hand currently has six blocks so we need to get rid of one. The {\LARGE\suo{2}\suo{4}} block is the weakest -- it is the only closed-wait protorun -- so we should get rid of this one. We should discard {\LARGE\suo{4}} first; if we draw {\LARGE\suo{3}} we will discard {\LARGE\suo{1}} to leave the possibility of {\jap pinfu}. If not, we discard {\LARGE\suo{2}} next, and then  {\LARGE\wan{3}}. 
\end{itembox}
%=====================

\vfill

\begin{itembox}[l]{Exercise 9}
What would you discard? \\
\vsp
How do you divide the hand into five tile blocks? 

\bp
\wan{2}\wan{4}\tong{7}\tong{8}\tong{9}\tong{9}\tong{9}\suo{2}\suo{2}\suo{6}\suo{7}\bai\bai~\wan{4}\\
\hfill\footnotesize{Draw~~~~~~~~~~~}
\ep
\end{itembox}
%=====================

\newpage

\begin{itembox}[r]{Answer 9}
\bmj{\Huge
$ 
\underbrace{\text{\wan{2}\wan{4}\wan{4}}}
\underbrace{\text{\tong{7}\tong{8}\tong{9}\tong{9}\tong{9}}}_{\text{\small 2}}
\underbrace{\text{\suo{2}\suo{2}}}
\underbrace{\text{\suo{6}\suo{7}}}
\underbrace{\text{\bai\bai}}
\nonumber
$
}\emj
We were planning to discard the {\LARGE\wan{2}\wan{4}} block because this was the weakest block among the six blocks in this hand. However, now that we drew another {\LARGE\wan{4}}, the {\LARGE\suo{2}\suo{2}} block is now the weakest. We thus discard {\LARGE\suo{2}}.
\end{itembox}
%=====================

\vfill

\begin{itembox}[l]{Exercise 10}
What would you discard? \\
\vsp
How do you divide the hand into five tile blocks? 

\bp
\wan{1}\wan{3}\wan{5}\wan{6}\wan{7}\tong{1}\tong{3}\suo{3}\suo{4}\suo{7}\suo{8}\bei\bei~\suo{5}\\
\hfill\footnotesize{Draw~~~~~~~~~~~}
\ep
\end{itembox}
%=====================

\newpage


\begin{itembox}[r]{Answer 10}
\bmj{\Huge
$ 
\underbrace{\text{\wan{1}\wan{3}}}
\underbrace{\text{\wan{5}\wan{6}\wan{7}}}
\underbrace{\text{\tong{1}\tong{3}}}
\underbrace{\text{\suo{3}\suo{4}\suo{5}}}
\underbrace{\text{\suo{7}\suo{8}}}
\underbrace{\text{\bei\bei}}
\nonumber
$
}\emj
The hand currently has six blocks so we need to get rid of one. Comparing the two closed-wait blocks {\LARGE\wan{1}\wan{3}} and {\LARGE\tong{1}\tong{3}}, the {\LARGE\wan{1}\wan{3}} block is more valuable because it is adjacent to a run. If we draw {\LARGE\wan{4}}, we will get a 3-way side-wait block. On the other hand, the {\LARGE\tong{1}\tong{3}} block will only become a 2-way side-wait block when we draw {\LARGE\tong{4}}. We should discard {\LARGE\tong{1}} first, not {\LARGE\tong{3}}, because if we draw {\LARGE\tong{4}} next, we will discard the {\LARGE\wan{1}\wan{3}} block.
\end{itembox}
%=====================

\vfill

\begin{itembox}[l]{Exercise 11}
What would you discard? \\
\vsp
How do you divide the hand into five tile blocks? 

\bp
\wan{5}\wan{7}\tong{1}\tong{1}\tong{3}\tong{7}\tong{7}\suo{1}\suo{1}\suo{4}\suo{5}\suo{6}\suo{7}~\suo{3}\\
\hfill\footnotesize{Draw~~~~~~~~~~~}
\ep
\end{itembox}
%=====================

\newpage


\begin{itembox}[r]{Answer 11}
\bmj{\Huge
$ 
\underbrace{\text{\wan{5}\wan{7}}}
\underbrace{\text{\tong{1}\tong{1}\tong{3}}}
\underbrace{\text{\tong{7}\tong{7}}}
\underbrace{\text{\suo{1}\suo{1}}}
\underbrace{\text{\suo{3}\suo{4}\suo{5}\suo{6}\suo{7}}}_{\text{\small 2}}
\nonumber
$
}\emj
Now that we have a 3-way side-wait block in {\jap souzu} (bamboos), we should get rid of one block. Comparing a closed-wait block {\LARGE\wan{5}\wan{7}} and two pairs {\LARGE\tong{7}\tong{7}} and {\LARGE\suo{1}\suo{1}}, we should value the closed-wait block. This is because the hand has three pairs already so we should get rid of one of them. Since we see a (remote) possibility of {\jap sanshoku} of 567, we should discard {\LARGE\suo{1}}. 
\end{itembox}
%=====================


\section{Building a block}

When a hand has fewer than five blocks, we need to build a new block, possibly from a floating tile we already have in the hand. In doing so, we should envision the kind of {\jap yaku} that the hand is going to have eventually. Consider the following hand. Suppose you are the dealer and this is East-1. What would you discard?

\bp
\wan{6}\wan{7}\wan{7}\wan{8}\tong{3}\tong{3}\tong{4}\tong{9}\suo{4}\suo{7}\suo{8}\suo{9}\zhong\bei
\ep

As usual, we will split the hand into blocks. Notice that the hand has at most four blocks only. 
\bmj{\Huge
$ 
\underbrace{\text{\wan{6}\wan{7}\wan{7}\wan{8}}}_{\text{\small 2}}
\underbrace{\text{\tong{3}\tong{3}\tong{4}}}
\text{\tong{9}}~\text{\suo{4}}
\underbrace{\text{\suo{7}\suo{8}\suo{9}}}
\text{\zhong~\bei}
\nonumber
$
}\emj \label{hand:head}
We should thus compare the four floating tiles {\LARGE\tong{9} \suo{4} \zhong ~\bei} in terms of their relative capabilities to grow into an independent block. Of these four tiles, {\LARGE\suo{4}} is the strongest candidate, because it can form a side-wait protorun with two kinds of tiles, \text{\suo{3}} and \text{\suo{5}}. Any simple tiles between 3 and 7 are a strong floating tile because of their ability to form a side-wait protorun. Terminals (1 and 9) will never become a side-wait protorun, and 2 and 8 can become a side-wait protorun when paired with only one kind of tiles (3 or 7). However, number tiles are still stronger than honor tiles because honor tiles can never form a run. 

\bigskip
We should thus choose between the two honor tiles, {\LARGE\zhong} and {\LARGE\bei}. Which one should we discard? Notice that this hand is clearly a {\jap pinfu} hand and that it is currently lacking the head. Since value tiles can never be the head of a {\jap pinfu} hand, we should discard {\LARGE\zhong} rather than {\LARGE\bei}. 

\bigskip
We may want to choose a discard from an existing block rather than discarding a floating tile in order to enhance the hand value. Consider the following hand. 
\bp
\wan{5}\wan{6}\wan{6}\wan{8}\tong{1}\tong{2}\tong{2}\tong{6}\suo{1}\suo{1}\suo{4}\suo{5}\suo{6}\suo{7}
\ep
From a pure perspective of tile efficiency, the discard choice should be either {\LARGE\suo{4} \suo{7}} or {\LARGE\tong{6}}, for discarding either of the three will maximize tile acceptance. The block configuration behind such a  decision is as follows. 
\bmj{\Huge
$ 
\underbrace{\text{\wan{5}\wan{6}}}
\underbrace{\text{\wan{6}\wan{8}}}
\underbrace{\text{\tong{1}\tong{2}\tong{2}}}
\text{\tong{6}}
\underbrace{\text{\suo{1}\suo{1}}}
\underbrace{\text{\suo{4}\suo{5}\suo{6}\suo{7}}}
\nonumber
$
}\emj
However, doing so makes it almost inevitable that the hand ends up having a low score and/or a bad wait. Alternatively, we can expect the stretched single shape {\LARGE\suo{4}\suo{5}\suo{6}\suo{7}} to produce two runs, {\LARGE\tong{6}} to form a run, and the tiles in {\jap manzu} (cracks) to produce one run, as follows. 

\bmj{\Huge
$ 
\underbrace{\text{\wan{5}\wan{6}\wan{6}\wan{8}}}
\underbrace{\text{\tong{1}\tong{2}\tong{2}}}
\underbrace{\text{\text{\tong{6}}}}
\text{\suo{1}\suo{1}}
\underbrace{\text{\suo{4}\suo{5}\suo{6}\suo{7}}}_{\text{\small 2}}
\nonumber
$
}\emj
We should thus discard the {\LARGE\tong{1}} for now, anticipating to discard the pair of {\LARGE\suo{1}} eventually. That way, we can expect to have {\jap tanyao}, {\jap pinfu}, and possibly {\jap sanshoku}. 

\vfill

\subsection*{Exercises: building a block}

\bigskip

\bigskip

\begin{itembox}[l]{Exercise 12}
What would you discard? \\
\vsp
How do you divide the hand into five tile blocks? 

\vspace{-30pt}
\bp
\hspace{-163pt}{\footnotesize\color{red!75!black} Red}\\ \vspace{-16pt}
\wan{1}\wan{3}\rfw\wan{8}\wan{9}\tong{3}\tong{4}\tong{4}\suo{2}\suo{2}\suo{3}\suo{5}\suo{6}\suo{7}
\ep
\end{itembox}
%=====================

\newpage

\begin{itembox}[r]{Answer 12}
\bmj{\Huge
$ 
\underbrace{\text{\wan{1}\wan{3}\rfw}}
\underbrace{\text{\wan{8}\wan{9}}}
\underbrace{\text{\tong{3}\tong{4}\tong{4}}}
\underbrace{\text{\suo{2}\suo{2}\suo{3}}}
\underbrace{\text{\suo{5}\suo{6}\suo{7}}}
\nonumber
$
}\emj
If we were to simply maximize tile acceptance, the discard choice should be either {\LARGE\wan{1}} or {\LARGE\rfw}. However, that would make the block in {\jap manzu} (cracks) too week. Breaking the {\LARGE\tong{3}\tong{4}\tong{4}} or {\LARGE\suo{2}\suo{2}\suo{3}} is not ideal, as these blocks are very strong. We should therefore discard the {\LARGE\wan{9}} to get rid of this edge-wait block. This will temporarily reduce the number of blocks from five to four, but we can expect to get back to five soon with this hand. 
\end{itembox}
%=====================

\vfill

\begin{itembox}[l]{Exercise 13}
What would you discard? \\
\vsp
How do you divide the hand into five tile blocks? 

\vspace{-30pt}
\bp
\hspace{-163pt}{\footnotesize\color{red!75!black} Red}\\ \vspace{-16pt}
\wan{1}\wan{3}\rfw\wan{5}\tong{3}\tong{4}\tong{4}\tong{5}\suo{2}\suo{2}\suo{3}\suo{5}\suo{6}\suo{7}
\ep
\end{itembox}
%=====================

\newpage

\begin{itembox}[r]{Answer 13}
\bmj{\Huge
$ 
\underbrace{\text{\wan{1}\wan{3}}}
\underbrace{\text{\rfw\wan{5}}}
\underbrace{\text{\tong{3}\tong{4}\tong{4}\tong{5}}}_{\text{\small 2}}
\underbrace{\text{\suo{2}\suo{2}\suo{3}}}
\underbrace{\text{\suo{5}\suo{6}\suo{7}}}
\nonumber
$
}\emj
Discarding {\LARGE\tong{4}} or {\LARGE\suo{2}} will make this hand 1-Away, so our choice is between these two options. Notice that the {\LARGE\wan{1}\wan{3}} block is weaker than the other four. As a back up, we should keep two {\LARGE\tong{4}} to maintain the bulging float block in {\jap pinzu} (dots) for now, hoping to get two runs out of it. If we draw {\LARGE\tong{2}-\tong{5}} or {\LARGE\tong{3}-\tong{6}} first, we will get rid of the {\LARGE\wan{1}\wan{3}} block. We should thus discard {\LARGE\suo{2}}. If we draw any of {\LARGE\suo{1}\suo{4}\wan{2}}, we should do insta-riichi. 
\end{itembox}
%=====================

\vfill

\begin{itembox}[l]{Exercise 14}
What would you discard? \\
\vsp
How do you divide the hand into five tile blocks? 

\vspace{-30pt}
\bp
\hspace{-128pt}{\footnotesize\color{red!75!black} Red}\\ \vspace{-16pt}
\wan{1}\wan{1}\wan{3}\rfw\wan{5}\tong{3}\tong{4}\tong{4}\tong{5}\suo{2}\suo{3}\suo{5}\suo{6}\suo{7}
\ep
\end{itembox}
%=====================

\newpage

\begin{itembox}[r]{Answer 14}
\bmj{\Huge
$ 
\underbrace{\text{\wan{1}\wan{1}\wan{3}\rfw\wan{5}}}_{\text{\small 2}}
\underbrace{\text{\tong{3}\tong{4}\tong{4}\tong{5}}}
\underbrace{\text{\suo{2}\suo{3}}}
\underbrace{\text{\suo{5}\suo{6}\suo{7}}}
\nonumber
$
}\emj
As we drew another {\LARGE\wan{1}}, the block in {\jap manzu} (cracks) is now a decent shape. This can become one group and the head with a draw of {\LARGE\wan{1} \wan{2} \wan{4} \wan{5}}. Therefore, we should discard  {\LARGE\tong{4}} to break the bulging float shape. 
\end{itembox}
%=====================

