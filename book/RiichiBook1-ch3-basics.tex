%~~~~~~~~~~~~~~~~~~~~~~~~~~~~~~~~~~~~~~~~~~~~~~~~~
% Riichi Book 1, Chapter 3: Basics
%~~~~~~~~~~~~~~~~~~~~~~~~~~~~~~~~~~~~~~~~~~~~~~~~~
\chapter{Les bases du mah-jong japonais}\label{ch:basic}
\thispagestyle{empty}

\section{Stratégies d’apprentissage}

Mahjong is a game of skill and luck. 
There is a set of strategy principles you can learn to improve your skills, but acquiring skills is neither necessary nor sufficient to win a game. On the contrary, with luck, an unskilled player can easily defeat strong players in mahjong. At least in the short run, game outcomes are governed more by luck than by skills.\footnote{An interesting question would be: how short is the ``short'' run here. That is, how many games do we need in order to discern a strong player from weak players? Studies show that we would need at least 100 games or so to have a reliable estimate of our skill levels. Given that EMA tournaments usually have only 8 games, winning at these tournaments requires quite a bit of luck.\index{european@EMA}}
However, learning strategy principles is crucial to improve your performance in the long run. Moreover, you will be able to enjoy the game in greater depth once you understand these principles. 

\bigskip
Because of the probabilistic nature of the game, making the best choice does not always lead to the best outcome. The best choices are those that lead to the best outcome, \emph{on average}. An evaluation of our choices thus requires a \emph{probabilistic} (i.e., statistical) assessment of different options. For example, consider the following hand. 
\vspace{5pt}
\begin{screen}
\bp
\wan{4}\wan{4}\tong{6}\tong{6}\tong{7}\suo{1}\suo{2}\suo{3}\suo{6}\suo{7}\suo{8}
\fa\fa\rfa
\ep
\vspace{-10pt}What would you discard? \vspace{-5pt}
\end{screen}

\bigskip
\noindent
This hand becomes ready to win if you discard {\LARGE\tong{6}} or {\LARGE\tong{7}}. Let's compare the two choices.
\bi\itemsep.1pt
\i Discard {\LARGE\tong{6}} \hfill $\Rightarrow$ you wait for {\LARGE\tong{5} \tong{8}} (2 kinds--8 tiles)
\i Discard {\LARGE\tong{7}} \hfill $\Rightarrow$ you wait for {\LARGE\wan{4}} {\LARGE\tong{6}} (2 kinds--4 tiles)
\ei
Which discard choice is better? Although both of the two choices yield a 2-way wait, waiting for {\LARGE\tong{5} \tong{8}} is much better than waiting for {\LARGE\wan{4}} {\LARGE\tong{6}}, probabilistically speaking. With the {\LARGE\tong{5} \tong{8}} wait, there are four tiles of {\LARGE\tong{5}} and another four tiles of {\LARGE\tong{8}} to win on, leaving at most eight winning tiles.\footnote{Of course, the number of winning tiles could be smaller than eight if some of them have already been discarded.} 
With the {\LARGE\wan{4}} {\LARGE\tong{6}} wait, on the other hand, you have already used up two tiles of {\LARGE\wan{4}} and two tiles of {\LARGE\tong{6}} yourself, leaving at most four winning tiles. It is clearly better to choose the {\LARGE\tong{5} \tong{8}} wait over the {\LARGE\wan{4}} {\LARGE\tong{6}} wait, because that will give you a higher probability of winning this hand. 

\bigskip
It is possible that, after you decided on the {\LARGE\tong{5} \tong{8}} wait, your opponents end up not discarding {\LARGE\tong{5}} or {\LARGE\tong{8}} at all, while discarding lots of {\LARGE\wan{4} \tong{6}}. This is the kind of things that will happen often in mahjong (or in any game of luck, for that matter). When things like this happen, do not think that you made a bad call; you didn't. You made the right choice, but you were just unlucky. When we experience this kind of bad luck, we just need to keep calm and carry on. 

\bigskip
Before discussing a practical method of maximizing tile efficiency in the next chapter, I will discuss some basic principles of tile efficiency in this chapter. In doing so, I introduce several key terms we use in later chapters. I will also provide the original Japanese term for each (shown in {\jap this font}). I do so because you may find these Japanese terms used in some online strategy discussions in English.

\section{Basic building blocks}

\subsection{Les tuiles}

Mahjong tiles can be classified into two categories --- number tiles and honor tiles.

\subsubsection{Number tiles}


\begin{itembox}[c]{Number tiles}
\centering
\begin{tabular}{l c c}
{Cracks (characters)} & {\jap Manzu} & {\Huge\wan{1}\wan{2}\wan{3}\wan{4}\wan{5}\wan{6}\wan{7}\wan{8}\wan{9}}\\ [\sep]
{Dots (circles)} & {\jap Pinzu} & {\Huge \tong{1}\tong{2}\tong{3}\tong{4}\tong{5}\tong{6}\tong{7}\tong{8}\tong{9}}\\ [\sep]
{Bamboos} & {\jap Souzu} & {\Huge\suo{1}\suo{2}\suo{3}\suo{4}\suo{5}\suo{6}\suo{7}\suo{8}\suo{9}}\\
\end{tabular}

\end{itembox}

\bigskip \noindent
We further classify number tiles into {\bf simples} ({\jap tanyao hai}; tiles between 2 and 8) and {\bf terminals} (1 and 9). They are differentiated because they serve different {\jap yaku} and generate different minipoints ({\jap fu}). 
	\index{simple tiles ({\jap tanyao hai})} \index{tanyao@{\jap tanyao hai} (simple tiles)}
	\index{terminal tiles}
	\index{fu@{\jap fu} (minipoint)} \index{minipoint ({\jap fu})}

\bigskip
It has become quite common to include some red five tiles. For example, most games on {\jap Tenhou} have one red five tile in each suit, {\LARGE \rfw~\rfd~\rfs}. These tiles are included in place of regular fives; we have three regular fives and one red five in each suit. Red fives are treated as {\jap dora} regardless of the {\jap dora} indicator. When a 4 in a given suit is the {\jap dora} indicator, the red five in that suit will be a double {\jap dora} tile. 

\vfill
\subsubsection{Honor tiles}
\index{honor tiles ({\jap jihai})} \index{jihai@{\jap jihai} (honor tiles)}

\begin{itembox}[c]{Honor tiles}
\centering \footnotesize
\begin{tabular}{l c c c c}
Dragon tiles & {\Huge\bai} & {\Huge\fa} & {\Huge\zhong}\\
& White ({\jap haku}) & Green ({\jap hatsu}) & Red ({\jap chun})\\ \\
Wind tiles &{\Huge\dong} &{\Huge\nan}& {\Huge\xi}& {\Huge\bei}\\
& East ({\jap ton}) &South ({\jap nan}) & West ({\jap sha}) &North ({\jap pei})\\
\end{tabular}
\end{itembox}

	\index{value tiles ({\jap fanpai; yakuhai})}
	\index{fanpai@{\jap fanpai} ({\jap yakuhai}; value tiles)}
	\index{yakuhai@{\jap yakuhai} ({\jap fanpai}; value tiles)}
	\index{valueless wind@valueless wind ({\jap otakaze})}
	\index{otakaze@{\jap otakaze} (valueless wind)}
\noindent Some honor tiles are {\bf value tiles} ({\jap fanpai / yakuhai}); we get one {\jap han} if we collect three identical value tiles. All dragon tiles are value tiles regardless of the round and seating. On the other hand, the value status of wind tiles depends on the round and the seating. East tiles are value tiles for everyone during the East round, and South tiles are value tiles for everyone during the South round. In addition, each player gets their own seating wind as a value tile. For example, West tiles are value tiles only for the West player, but they are valueless wind tiles ({\jap otakaze}) for other players.

\vfill
\subsection{Les groupes ({\jap mentsu})}
\index{run ({\jap shuntsu}; chow)}
\index{chow ({\jap shuntsu}; run)}
\index{shuntsu@{\jap shuntsu} (run; chow)}
\index{set ({\jap kotsu}; pung)}
\index{pung ({\jap kotsu}; set)}
\index{kotsu@{\jap kotsu} (set; pung)}
\index{mentsu@{\jap mentsu} (group)}
\index{group ({\jap mentsu})}
\index{kan@{\jap kan} (kong)} \index{kantsu@{\jap kantsu} (quad)}
\index{quad@quad ({\jap kantsu})}

One of the major goals in playing mahjong is to win a hand.\footnote{Another important goal is not to deal into an opponent's hand. See Chapter \ref{ch:defense} for discussions of defense strategies. However, the most important goal of all is to win a game. Winning a hand and playing defense are merely two means to this end. See Chapter \ref{ch:grand} for more discussions of this.}
To win a standard hand, we need to complete four groups ({\jap mentsu}) and one head ({\jap atama}; final pair).\footnote{There are three exceptions to this; {\jap chiitoitsu} (Seven Pairs), {\jap kokushi musou} (Thirteen Orphans), and {\jap nagashi mangan} (All Terminals and Honors Discard) do not require four groups and one head.} 
Groups can be classified into two kinds --- {\bf run} and {\bf set}.\footnote{EMA rules refer to run as ``chow'' and set as ``pung.'' I realize that my use of different terminology here might be confusing at first, but I hope you will get used to it soon.}
\bi
\i Run ({\jap shuntsu}; chow / sequence) is a set of three consecutive number tiles: e.g., {\LARGE \wan{7}\wan{8}\wan{9}}, {\LARGE \suo{3}\suo{4}\suo{5}}. 
\i Set ({\jap kotsu}; pung / triplet) is a set of three identical tiles: e.g., {\LARGE \wan{2}\wan{2}\wan{2}}, {\LARGE \fa\fa\fa}.\footnote{Technically speaking, there is a third type of groups, namely quad ({\jap kantsu}; kong), a set of four identical tiles. We treat quads as a variant of sets. See Section \ref{sec:kong} for discussions on this.} 
\ei
 

\subsection{Ready and $n$-away}
	\index{ready ({\jap tenpai})} \index{tenpai@{\jap tenpai} (ready)}
	\index{1-away (1-{\jap shanten})}
	\index{2-away (2-{\jap shanten})}
	\index{3-away (3-{\jap shanten})}
	\index{ukeire@{\jap ukeire} (tile acceptance)} 
	\index{tile acceptance ({\jap ukeire})}
We say a hand is {\bf ready} ({\jap tenpai}) when the hand can be complete with one more tile. For example, the following hand is ready. 
\bigskip
\begin{itembox}[r]{Ready hand}
\bp
\tong{3}\tong{4}\tong{5}\tong{8}\tong{8}\suo{2}\suo{3}\suo{6}\suo{7}\suo{8}\zhong\zhong\zhong
\ep
\end{itembox}
This hand becomes complete with either {\LARGE\suo{1}} or {\LARGE\suo{4}}. We say that this hand waits for {\LARGE\suo{1} \suo{4}}.

\bigskip

We say a hand is {\bf 1-away from ready} (1-{\jap shanten}) when the hand can become ready with one more tile. For example, the following hand is 1-away from ready. 
\bigskip
\begin{itembox}[r]{1-away hand}
\bp
\tong{3}\tong{5}\tong{5}\tong{8}\tong{8}\suo{2}\suo{3}\suo{6}\suo{7}\suo{8}\zhong\zhong\zhong
\ep
\end{itembox}
This hand becomes ready if you draw any of {\LARGE\tong{4}\tong{5}\tong{8}\suo{1}\suo{4}}. We say this hand accepts {\LARGE\tong{4}\tong{5}\tong{8}\suo{1}\suo{4}} (5 kinds--16 tiles) as any of them can make this hand advance from 1-away to ready. {\bf Tile acceptance} ({\jap ukeire}) refers to the kinds and the number of tiles a hand can accept. Other things being equal, having a 1-away hand with greater tile acceptance is better than having one with smaller tile acceptance. 

\bigskip
More generally, we say a hand is $n$-away from ready ($n$-{\jap shanten}) when the hand can be ready with $n$ more steps. For example, the following hand is 2-away from ready. 
\bigskip
\begin{itembox}[r]{2-away hand}
\bp
\tong{3}\tong{5}\tong{5}\tong{8}\tong{8}\suo{2}\suo{3}\suo{6}\suo{7}\suo{7}\zhong\zhong\zhong
\ep
\end{itembox} \vspace{5pt}
\noindent
This hand accepts all the tiles that the 1-away hand above accepts ({\LARGE\tong{4}\tong{5}\tong{8}\suo{1}\suo{4}}), plus seven additional kinds of tiles {\LARGE\tong{3}\suo{2}\suo{3}\suo{5}\suo{6}\suo{7}\suo{8}}.\footnote{\tong{3}\suo{2}\suo{3}\suo{6} will make this hand 1-away for {\jap chiitoitsu} (Seven Pairs).} 
The hand will become 1-away if any of these tiles gets drawn. 

\bigskip
A hand can also be 3-away, 4-away, 5-away, or 6-away from ready.\footnote{6-away happens when a hand has no pair, in which case it takes 6 more tiles to make it ready for {\jap chiitoitsu}.}  
In practice, however, there is not much point in distinguishing 3-away hands from 4-away (or worse) hands. You thus need to be able to distinguish between four kinds of hands --- ready hands, 1-away hands, 2-away hands, and 3-away or worse hands. 

\subsubsection{Tile acceptance shrinkage} 
As $n$ gets smaller and the hand gets closer to completion, the kinds and the number of tiles it can accept will necessarily get smaller. Consider the three stages of a hand we have seen above.
\bi
\i When 2-away, it accepts: {\LARGE\tong{3}\tong{4}\tong{5}\tong{8}\suo{1}\suo{2}\suo{3}\suo{4}\suo{5}\suo{6}\suo{7}\suo{8}}.
\i When 1-away, it accepts: {\LARGE\tong{4}\tong{5}\tong{8}\suo{1}\suo{4}}.
\i When ready, it waits for: {\LARGE\suo{1}\suo{4}}.
\ei
Tile acceptance is minimized when the hand is ready. Note also that it is \emph{virtually} minimized when it is 1-away. This is because with a ready hand you can utilize not only the tiles you draw but also the tiles discarded by others to complete the hand. With $n$-away hands, however, you have to rely (almost) solely on the tiles you draw yourself to advance your hand.\footnote{Melding (calling {\jap pon} / {\jap chii}) is not always possible. For example, the 2-away hand above can accept \suo{3} if you draw one, but you can neither {\jap pon} nor {\jap chii} \suo{3}.} 
Therefore, in choosing a discard from a 2-away hand, we should try not to make for a 1-away hand with too small tile acceptance.

\subsubsection{Advancing your hand}

To win a hand, we need to advance our hand by reducing the $n$ of an $n$-away hand until it is ready.
When a hand is 2-away, we should aim to make the hand 1-away. 
When a hand is 1-away, we should aim to make the hand ready. 
For example, consider the following hand. 
\begin{itembox}[r]{2-away vs. 1-away}
\bp
\wan{5}\wan{6}\wan{7}\wan{7}\tong{3}\tong{4}\suo{4}\suo{5}\suo{5}\suo{7}\suo{9}\fa\fa\fa
\ep
\vspace{-10pt}What would you discard? \vspace{-5pt}
\end{itembox}
\noindent
Discarding {\LARGE\suo{9}} makes the hand 2-away, whereas discarding either {\LARGE\wan{7}} or {\LARGE\suo{4}} makes the hand 1-away. You should thus discard {\LARGE\wan{7}} or {\LARGE\suo{4}} to make the hand 1-away. 
Reverting a 1-away hand to 2-away makes sense only in some exceptional cases where tile acceptance at 1-away becomes unbearably small (i.e., fewer than 2 kinds).
With this hand, the hand will be able to accept {\LARGE\tong{2}\tong{5}\suo{8}} (3 kinds--12 tiles) when it becomes 1-away.

\subsection{Les protosuites ({\jap taatsu})} \label{sec:protorun}
	\index{protorun ({\jap taatsu})} \index{taatsu@{\jap taatsu} (protorun)}
	\index{waits!side wait} \index{waits!closed wait} \index{waits!edge wait}

Of the two kinds of groups, it is easier to complete a run than to complete a set. There are only four identical tiles, and completing a set requires that you collect three out of the four identical tiles. 
Therefore, we usually prioritize runs over sets in advancing a hand. 

\bigskip
A pair of tiles that can become a run with one more tile is called a {\bf protorun} ({\jap taatsu}). There are three types of protoruns, summarized in Table \ref{tbl:proto}. 
\bigskip

{\begin{table}[h!]\centering \small \captionsetup{font=footnotesize}
\caption{Types de protosuites} \label{tbl:proto}
\begin{tabular}{l l c c l} 
\toprule
Name & Japanese & Example & Wait & Acceptance\\ 
\midrule 
side wait & {\jap ryanmen} & {\LARGE \wan{3}\wan{4}} & {\LARGE \wan{2}-\wan{5}} & 2 kinds--8 tiles\\ [\sep]
closed wait & {\jap kanchan} & {\LARGE \tong{2}\tong{4}} & {\LARGE \tong{3}} & 1 kind--4 tiles\\ [\sep]
edge wait & {\jap penchan} & {\LARGE \suo{8}\suo{9}} & {\LARGE \suo{7}} & 1 kind--4 tiles\\ [\sep]
\bottomrule
\end{tabular}
\end{table}}

As we can see in the table, a {\bf side-wait} ({\jap ryanmen}) protorun can accept twice as many tiles as a {\bf closed-wait} ({\jap kanchan}) protorun or an {\bf edge-wait} ({\jap penchan}) protorun can. Therefore, building side-wait protoruns is the key to advancing a hand. Winning tiles of side-wait protoruns are often denoted with a hyphen in the middle, such as {\LARGE \suo{1}-\suo{4}} or {\LARGE \tong{5}-\tong{8}}.

\bigskip
\subsubsection*{Closed wait vs. edge wait}
There is no difference in the kinds and the number of tiles accepted by closed-wait and edge-wait protoruns; they both accept 1 kind--4 tiles. However, closed-wait protoruns are superior to edge-wait ones because they can more easily evolve into a side-wait protorun. 

\bigskip
A closed-wait protorun can evolve into a side-wait protorun in just one step. For example, a protorun {\LARGE\tong{1}\tong{3}} can become a side-wait one if you draw {\LARGE\tong{4}} and discard {\LARGE\tong{1}}. 
\begin{center}
{\LARGE \tong{1}\tong{3} ~~$\Rightarrow$~~ \tong{3}\tong{4}}\\
draw {\Large\tong{4}}
\end{center}
On the other hand, it requires two steps for an edge-wait protorun to evolve into a side-wait protorun. For example, a protorun {\LARGE\tong{8}\tong{9}} can become a side-wait one if you draw {\LARGE\tong{6}} first and then {\LARGE\tong{5}}. 
\begin{center}
{\LARGE \tong{8}\tong{9} ~~$\Rightarrow$~~ \tong{6}\tong{8} ~~ $\Rightarrow$ ~~ 
\tong{5}\tong{6}}\\
draw {\Large\tong{6}} ~~~~~~~~~~~~ draw {\Large\tong{5}}
\end{center}

\bigskip

\color{MyRed}
\begin{itembox}[c]{Value ranking of protoruns}
\normalcolor
\centering
side wait > closed wait > edge wait
\end{itembox}
\normalcolor

\bigskip

\subsubsection{Tile versatility} \label{sec:versatility}

Some tiles are more versatile than others. 
For example, number tiles are more versatile than honor tiles because honor tiles can never form a run. Moreover, we can rank order the versatility of number tiles based on the types of protoruns they can form.

\bigskip
Number tiles between 3 and 7 are the most versatile. This is because each of them can form a protorun with four kinds of number tiles. For example, {\LARGE\tong{3}} can form a protorun with {\LARGE\tong{1}}, {\LARGE\tong{2}}, {\LARGE\tong{4}}, and {\LARGE\tong{5}}. Two out of the four resulting protoruns will be side wait.

\bigskip
2 and 8 are less versatile. They can form a protorun with only three kinds of number tiles. For example, {\LARGE\tong{2}} can form a protorun with {\LARGE\tong{1}}, {\LARGE\tong{3}}, and {\LARGE\tong{4}}. Only one out of the three resulting protoruns is side wait. 

\bigskip
Terminals (1 and 9) are the least versatile. They can form a protorun with only two kinds of tiles. For example, {\LARGE\tong{1}} can form a protorun only with {\LARGE\tong{2}} and {\LARGE\tong{3}}. Neither of the two resulting protoruns is side wait. 

\color{MyRed}
\begin{itembox}[c]{Versatility ranking of tiles}
\centering\normalcolor
3--7 tiles > 2, 8 tiles > 1, 9 tiles > honor tiles
\end{itembox}
\normalcolor
\bigskip
Applying the same logic, we can also rank order the versatility of closed-wait protoruns. 
For example, a closed-wait protorun {\LARGE\tong{1}\tong{3}} can become a side-wait one only if we draw {\LARGE\tong{4}}. Likewise, a closed-wait protorun {\LARGE\tong{2}\tong{4}} can become a side-wait one only if we draw {\LARGE\tong{5}}. However, a closed-wait protorun {\LARGE\tong{3}\tong{5}} can become a side-wait one if we draw {\LARGE\tong{2}} or {\LARGE\tong{6}}. Clearly, {\LARGE\tong{3}\tong{5}} is more versatile than {\LARGE\tong{1}\tong{3}} or {\LARGE\tong{2}\tong{4}}. 

\color{MyRed}
\begin{itembox}[c]{Versatility ranking of closed-wait protoruns}
\centering\normalcolor
35, 46, 57 > 13, 24, 68, 79
\end{itembox} 
\normalcolor
\bigskip

\subsection{Les paires ({\jap toitsu})}\label{sec:closevers}
\index{pair ({\jap toitsu})} \index{toitsu@{\jap toitsu} (pair)}

A set of two identical tiles is called a {\bf pair} ({\jap toitsu}). 
Pairs can perform several different roles. A pair can be the head (final pair) of a hand, a protoset (a candidate for a set), or a component of {\jap chiitoitsu} (Seven Pairs). 

\bigskip
\subsubsection{Building the head}
Any hand --- including Thirteen Orphans and Seven Pairs --- requires the head to be complete. Since building the head is much easier than building a group, we usually don't worry too much about the head. For example, consider the following hand. 
\bigskip
\begin{itembox}[r]{Hand with no head}
\bp
\wan{7}\wan{8}\wan{9}\tong{2}\tong{3}\tong{4}\tong{5}\tong{6}\tong{7}\suo{3}\suo{4}\suo{5}\suo{7}
\ep
\end{itembox}
This hand currently lacks the head and the wait is not very good. The hand is complete only with {\LARGE\suo{7}} (1 kind--3 tiles).
However, if we draw any of {\LARGE\wan{6}\wan{9}\tong{1}\tong{2}\tong{4}\tong{5}\tong{7}\tong{8}\suo{2}\suo{3}\suo{5}\suo{6}} (12 kinds--41 tiles), the wait will be significantly improved. 
For example, if we draw {\LARGE\tong{5}} and discard {\LARGE\suo{7}}, the hand becomes:
\bp
\wan{7}\wan{8}\wan{9}\tong{2}\tong{3}\tong{4}\tong{5}\tong{5}\tong{6}\tong{7}\suo{3}\suo{4}\suo{5}
\ep
This hand is now waiting for {\LARGE\tong{2} \tong{5}-\tong{8}} (3 kinds--9 tiles). 
When a hand is missing the head, it is often the case that the wait gets significantly improved quite easily. 

\subsection{Parlons de paires et de brelans} \label{sec:2pairs}
Another important role a pair can play is to work as a candidate for a set. 
Especially when a hand has two pairs, we can count on one of the two pairs to become the head while the other becomes a set. In other words, the value of pairs is maximized when there are two (and only two) pairs in a hand. Let's see why this is the case by comparing hands with one, two, and three pairs. 

%\newpage
\begin{itembox}[r]{1. Hand with one pair}
\bp
\wan{1}\wan{3}\wan{5}\tong{5}\tong{6}\tong{7}\tong{8}\tong{9}\suo{2}\suo{2}\suo{6}\suo{7}\bei
\ep
\end{itembox}
\noindent
This 2-away hand has one pair: {\LARGE\suo{2}\suo{2}}. This pair is not very useful as a candidate for a set for two reasons. First, if we draw another {\LARGE\suo{2}}, we will complete a set but then we will lose the head at the same time. The hand will still be 2-away from ready after all. Second, the probability of drawing another {\LARGE\suo{2}} is not very high because there are only two tiles left. 

\bigskip
What if a hand has two pairs? Suppose we drew {\LARGE\wan{1}} and discarded {\LARGE\bei}, as follows.

\bigskip
\begin{itembox}[r]{2. Hand with two pairs}
\bp
\wan{1}\wan{1}\wan{3}\wan{5}\tong{5}\tong{6}\tong{7}\tong{8}\tong{9}\suo{2}\suo{2}\suo{6}\suo{7}
\ep
\end{itembox}
\noindent This hand is also 2-away, but it has two pairs: {\LARGE\wan{1}\wan{1}} and {\LARGE\suo{2}\suo{2}}. 
Each of these pairs is now functioning as an effective candidate for a set. Whenever one pair becomes a set, the other pair becomes the head. Drawing {\LARGE\wan{1}} or {\LARGE\suo{2}} will advance this hand from 2-away to 1-away. 

\bigskip
Moreover, whereas the hand with one pair was able to accept two tiles of {\LARGE\suo{2}}, the hand with two pairs can accept four tiles (two of {\LARGE\suo{2}} and two of {\LARGE\wan{1}}). The probability of drawing any one of four tiles is obviously higher than the probability of drawing any one of two tiles. 
In general, for each additional pair in a hand, tile acceptance increases by two. 

\bigskip
What if a hand has three pairs? Suppose we draw {\LARGE\suo{6}}, as follows. 

\bigskip
\begin{itembox}[r]{3. Hand with three pairs}
\bp
\wan{1}\wan{1}\wan{3}\wan{5}\tong{5}\tong{6}\tong{7}\tong{8}\tong{9}\suo{2}\suo{2}\suo{6}\suo{7}~\suo{6}\\
\hfill\footnotesize{Draw~~~~~~~~~~~}
\ep
\end{itembox}
\noindent If we keep the second {\LARGE\suo{6}} and discard the {\LARGE\suo{7}} or the {\LARGE\wan{5}}, the hand has three pairs. However, keeping three pairs in a hand is inefficient. Recall that each additional pair increases tile acceptance by two tiles. In this case, keeping a pair of {\LARGE\suo{6}} means that the hand can accept two additional tiles of {\LARGE\suo{6}}. However, doing so comes with a cost. Keeping three pairs by discarding the {\LARGE\suo{7}} means the hand can no longer accept {\LARGE\suo{5} \suo{8}} (2 kinds--8 tiles). The net tile acceptance gain will be negative ($2-8 = -6$). Similarly, keeping three pairs by discarding the {\LARGE\wan{5}} means the hand can no longer accept {\LARGE\wan{4}} (4 tiles). 
%Again, the net tile acceptance gain is negative. 
Therefore, discarding a {\LARGE\suo{6}} to maintain two pairs is the most efficient. 

\bigskip
What we have seen so far is generalizable beyond the current examples. As long as we intend to keep the hand closed (i.e., not calling {\jap pon} or {\jap chii}), we should avoid having three pairs in a hand. Having three pairs makes for the weakest form, whereas having two pairs makes for the strongest form.\footnote{What if there are four or more pairs? Whenever a hand has four pairs, it is 2-away from ready for {\jap chiitoitsu} (Seven Pairs). It may be faster to pursue {\jap chiitoitsu} than pursuing a standard hand in such cases.}

\bigskip

\color{MyRed}
\begin{itembox}[c]{Value of pairs: closed hand}
\centering\normalcolor
2 pairs > 1 pair, 4 pairs > 3 pairs
\end{itembox}\normalcolor

\bigskip

\subsubsection{Main ouverte}

There is an important caveat to the above rule. 
When we intend to call {\jap pon}, having three pairs is actually better than having two pairs. This is because the hand will become a two-pair hand after we call {\jap pon} once.
For example, consider the following hand. 
\begin{itembox}[r]{Two pairs vs. three pairs}
\bp
\wan{1}\wan{1}\wan{3}\wan{5}\tong{4}\tong{5}\suo{2}\suo{4}\suo{4}\suo{7}\suo{8}\suo{9}\zhong\zhong~\zhong\\
\hfill\footnotesize{{\jap Dora}~~~~~~~~}
\ep
\vspace{-17pt}What would you discard? \vspace{-5pt}
\end{itembox}
\noindent
We would definitely intend to call {\jap pon} on {\LARGE\zhong}. Anticipating that, we should discard {\LARGE\wan{5}} to keep three pairs in this case rather than discarding {\LARGE\wan{1}} to have two pairs. After calling {\jap pon} on {\LARGE\zhong}, we will have a choice between discarding  {\LARGE\wan{3}} or {\LARGE\suo{2}}.
\bp
\wan{1}\wan{1}\tong{4}\tong{5}\suo{2}\suo{4}\suo{4}\suo{7}\suo{8}\suo{9}~\zhong\rzhong\zhong\\
\wan{1}\wan{1}\wan{3}\tong{4}\tong{5}\suo{4}\suo{4}\suo{7}\suo{8}\suo{9}~\zhong\rzhong\zhong
\ep
In either case, the hand will have two pairs after calling {\jap pon}. 

\bigskip

\color{MyRed}
\begin{itembox}[c]{Value of pairs: open hand}
\centering\normalcolor
3 pairs > 2 pairs
\end{itembox}\normalcolor

\bigskip

\subsection{Perfect $n$-away}\label{sec:perfect}

\subsubsection{Perfect 1-away} 
	\index{1-away (1-{\jap shanten})!perfect 1-away}
When a 1-away hand has two side-wait protoruns and two pairs, it is called {\bf perfect 1-away}. 
\bigskip
\begin{itembox}[r]{Perfect 1-away}
\bp
\wan{2}\wan{3}\wan{3}\tong{1}\tong{2}\tong{3}\tong{6}\tong{6}\suo{6}\suo{7}\bai\bai\bai\\
\ep
\end{itembox}
The hand above is an example of perfect 1-away. It is called ``perfect'' because this hand can become ready either by calling {\jap chii}, calling {\jap pon}, or drawing a tile to complete a run or a set, and no matter how a hand becomes ready, you will \emph{always} have the option to choose side wait as the final wait.

\subsubsection{Perfect 2-away}
	\index{2-away (2-{\jap shanten})!perfect 2-away}
One step prior to achieving perfect 1-away, we may get a perfect 2-away hand. Perfect 2-away is made up with three side-wait protoruns and three pairs, as follows.
\bigskip
\begin{itembox}[r]{Perfect 2-away}
\bp
\wan{2}\wan{3}\wan{3}\tong{2}\tong{3}\tong{6}\tong{6}\suo{6}\suo{7}\suo{7}\bai\bai\bai\\
\ep
\end{itembox}
When a perfect 2-away hand becomes 1-away, it can always be perfect 1-away (unless you choose not to, for some reason). However, not all perfect 1-away hands evolve from a perfect 2-away hand. 

\newpage
\subsection{Putting things all together: an example}\label{sec:ptat}

Let's see some hand examples that illustrate how we can apply the tile efficiency logics we have learned so far. Consider the following 2-away hand. 
\begin{itembox}[r]{Advancing a hand 1}
\bp
\wan{5}\wan{5}\wan{7}\suo{2}\suo{4}\suo{8}\suo{8}\suo{9}\tong{2}\tong{3}\tong{7}\tong{8}\tong{9}~\suo{2}\\
\hfill\footnotesize{Draw~~~~~~~~~~~}
\ep
\vspace{-17pt}What would you discard? \vspace{-5pt}
\end{itembox}
\noindent
The hand now has three pairs, and we should avoid it. In order to reduce the number of pairs in this hand from three to two, our discard candidates should be {\LARGE\wan{5}}, {\LARGE\suo{2}}, or {\LARGE\suo{8}}. Which one should we choose? 

\bigskip
Recall that a closed-wait protorun of 57 is stronger than a closed-wait protorun of 24 or an edge-wait protorun of 89. Therefore, it is OK to cut down the {\LARGE\wan{5}\wan{5}\wan{7}} shape to {\LARGE\wan{5}\wan{7}} by discarding {\LARGE\wan{5}}. This is because {\LARGE\wan{5}\wan{7}} can become a side-wait protorun relatively easily. On the other hand, the {\LARGE\suo{2}\suo{2}\suo{4}} shape and the {\LARGE\suo{8}\suo{8}\suo{9}} shape are both weak; the first can become a side-wait protorun only if we draw {\LARGE\suo{5}}, and the second one will never become a side-wait protorun in one step. Therefore, both {\LARGE\suo{2}\suo{2}\suo{4}} and {\LARGE\suo{8}\suo{8}\suo{9}} should be kept as a candidate for the head or a group rather than making them into weak closed-wait protoruns.

\bigskip
Let's say we discard {\LARGE\wan{5}}, and then we draw {\LARGE\wan{8}}, resulting in the following hand. 
\begin{itembox}[r]{Advancing a hand 2}
\bp
\wan{5}\wan{7}\suo{2}\suo{2}\suo{4}\suo{8}\suo{8}\suo{9}\tong{2}\tong{3}\tong{7}\tong{8}\tong{9}~\wan{8}\\
\hfill\footnotesize{Draw~~~~~~~~~~~}
\ep
\vspace{-17pt}What would you discard? \vspace{-5pt}
\end{itembox}
\noindent
Now that we have a side-wait protorun {\LARGE\wan{7}\wan{8}}, we should discard {\LARGE\wan{5}}. 

\bigskip
Let's say we draw {\LARGE\suo{7}}, resulting in the following hand. 
\begin{itembox}[r]{Advancing a hand 3}
\bp
\wan{7}\wan{8}\suo{2}\suo{2}\suo{4}\suo{8}\suo{8}\suo{9}\tong{2}\tong{3}\tong{7}\tong{8}\tong{9}~\suo{7}\\
\hfill\footnotesize{Draw~~~~~~~~~~~}
\ep
\vspace{-17pt}What would you discard? \vspace{-5pt}
\end{itembox}
\noindent
This hand is now 1-away from ready, and our discard choice is between {\LARGE\suo{4}} and {\LARGE\suo{8}}. Both tiles are equally useless from our perspective, and so we will eventually discard them both. The question is which one we should discard first. Recall that a 4 is more versatile than an 8. This means that {\LARGE\suo{4}} in this hand may later become dangerous for the opponents; we should thus discard {\LARGE\suo{4}} now rather than later. 

\bigskip
Let's say we draw {\LARGE\tong{4}} after that, resulting in the following hand.
\begin{itembox}[r]{Advancing a hand 4}
\bp
\wan{7}\wan{8}\suo{2}\suo{2}\suo{7}\suo{8}\suo{8}\suo{9}\tong{2}\tong{3}\tong{7}\tong{8}\tong{9}~\tong{4}\\
\hfill\footnotesize{Draw~~~~~~~~~~~}
\ep
\vspace{-17pt}What would you discard? \vspace{-5pt}
\end{itembox}
\noindent
The hand is now ready. We should discard {\LARGE\suo{8}} and call riichi. If we win on {\LARGE\wan{9}}, we can claim riichi, {\jap pinfu}, and {\jap sanshoku} (Mixed Triple Chow), giving us 7700 points.\footnote{We will discuss scoring and {\jap yaku} more extensively in later chapters.}

\newpage
\section{Complex shapes}
The three basic types of tile blocks we have covered so far --- groups (runs and sets), protoruns (side wait, closed wait, and edge wait), and pairs --- form the basis of any standard mahjong hands.\footnote{Standard hands are those with four groups and one head. Non-standard hands are {\jap chiitoitsu} (Seven Pairs) and {\jap kokushi musou} (Thirteen Orphans).}  
When a hand has some tiles that do not constitute any of these three shapes, we treat them as {\bf floating tiles}. For example, {\LARGE\wan{5}} and {\LARGE\suo{6}} in the following hand are both floating tiles. 
\bigskip
\begin{itembox}[r]{Hand with floating tiles}
\bp
\wan{5}\wan{8}\wan{9}\tong{2}\tong{3}\tong{4}\tong{5}\tong{6}\tong{7}\suo{1}\suo{2}\suo{3}\suo{6}
\ep
\end{itembox}

\bigskip
In addition to these basic blocks, we often come across complex shapes that are made up of two or more groups, protoruns, pairs, and floating tiles combined. It is useful to comprehend such complex shapes as they are rather than breaking them up into smaller parts. We will discuss three-tile complex shapes and four-tile complex shapes in turn. 

\subsection{Three-tile complex shapes} 

There are two kinds of three-tile complex shapes --- double closed shape and protorun plus one shape.

\vfill
\subsubsection{Double closed ({\jap ryankan}) shape} \label{sec:ryankan}
	\index{double closed shape@double closed ({\jap ryankan}) shape} 
	\index{ryankan@{\jap ryankan} (double closed) shape}
When two closed-wait protoruns are combined, we have a {\bf double closed} ({\jap ryankan}) shape. 
There are five different patterns in each suit, as follows.
\bp
\suo{1}\suo{3}\suo{5}~~\suo{2}\suo{4}\suo{6}~~\suo{3}\suo{5}\suo{7}\\
\suo{4}\suo{6}\suo{8}~~\suo{5}\suo{7}\suo{9}
\ep
Each shape accepts as many as 2 kinds--8 tiles. For example, {\LARGE\suo{1}\suo{3}\suo{5}} accepts {\LARGE\suo{2}} (4 tiles) and {\LARGE\suo{4}} (4 tiles). This is twice as many as the number of tiles an isolated closed-wait protorun can accept. 

\bigskip
Sometimes a double closed shape is embedded within a tile block, making it difficult to detect it. 
For example, consider the following 1-away hand. 
\begin{itembox}[r]{Hand with a double closed shape}
\vspace{-5pt}
\bp
\hspace{-202pt}{\footnotesize\color{red!75!black} Red}\\ \vspace{-16pt}
\wan{4}\rfw\wan{5}\wan{7}\wan{8}\wan{9}\tong{4}\tong{5}\tong{6}\tong{6}\tong{7}\suo{7}\suo{7}~\tong{2}\\
\hfill\footnotesize{Draw~~~~~~~~~~~}
\ep
\vspace{-15pt}What would you discard? \vspace{-5pt}
\end{itembox}
\noindent
Before drawing {\LARGE\tong{2}}, the hand was already in a very good shape. It was perfect 1-away, accepting any of {\LARGE\wan{3}\wan{5}\wan{6}\tong{5}\tong{8}\suo{7}} (6 kinds--19 tiles). The question is whether we should keep {\LARGE\tong{2}} and discard {\LARGE\wan{5}} instead. \index{1-away (1-{\jap shanten})!perfect 1-away}

\bigskip
Notice that, if we keep {\LARGE\tong{2}}, we have a double closed shape {\LARGE\tong{2}\tong{4}\tong{6}}. This is because the block {\LARGE\tong{2}\tong{4}\tong{5}\tong{6}\tong{6}\tong{7}} can be split into {\LARGE\tong{2}\tong{4}\tong{6}} and {\LARGE\tong{5}\tong{6}\tong{7}}. 
If we keep {\LARGE\tong{2}} and discard {\LARGE\wan{5}}, the hand is still 1-away from ready, accepting {\LARGE\wan{3}\wan{6}\tong{3}\tong{5}\tong{8}} (5 kinds--19 tiles). The benefit of discarding {\LARGE\wan{5}} to keep the double closed shape is that the hand can \emph{always} be {\jap pinfu} when it is ready. On the other hand, discarding {\LARGE\tong{2}} means that the hand may become a {\jap yaku}-less hand when drawing {\LARGE\wan{5}} or {\LARGE\suo{7}}.

\bigskip
Double closed shapes are particularly useful when a hand is relatively far from ready (2-away or worse). As a hand advances, however, its usefulness diminishes because this block requires \underline{three} (not two) tiles even though it is not a complete group. 
Moreover, it will ultimately become a single closed-wait protorun when this block remains incomplete when the hand is ready. 
Therefore, we should not rely too much on a double closed shape. 
For example, consider the following two hands. 
\bp
\wan{1}\wan{1}\wan{3}\wan{4}\wan{5}\tong{6}\tong{7}\tong{8}\suo{2}\suo{4}\suo{6}\zhong\zhong~\wan{2}\\
\hfill\footnotesize{Draw~~~~~~~~~~~~~~~}
\ep 
\vspace{-20pt}
\bp
\wan{3}\wan{4}\wan{7}\wan{8}\wan{9}\tong{3}\tong{4}\tong{5}\tong{8}\tong{8}\suo{2}\suo{4}\suo{6}~\tong{7}\\
\hfill\footnotesize{Draw~~~~~~~~~~~~~~~}
\ep

Both hands are 1-away from ready and both contain a double closed shape in {\jap souzu} (bamboos) tiles. Maintaining the double closed shape in these cases will not be ideal. 
It is true that, if the hand becomes ready by drawing {\LARGE\suo{3}} or  {\LARGE\suo{5}} first, each of the hands makes for a good-wait ready hand. However, if the first hand becomes ready by calling {\jap pon} on {\LARGE\zhong} or the second hand becomes ready by drawing {\LARGE\wan{2}} or {\LARGE\wan{5}} first, they only make for a closed-wait ready hand. 

\bigskip
Therefore, when we draw a tile next to the head, creating a side-wait protorun, we should keep it and break the double closed shape instead. In the first example above, as we draw {\LARGE\wan{2}} that creates a side-wait protorun {\LARGE\wan{4}\wan{5}}, we should keep it and discard the {\LARGE\suo{2}} instead. In the second example above, as we draw {\LARGE\tong{7}} that creates a side-wait protorun {\LARGE\tong{7}\tong{8}}, we should keep it and discard {\LARGE\suo{2}} instead. 

\subsubsection{Les formes « protogroupe + 1 »}

As we saw with the first example in Section \ref{sec:ptat}, we often come across a tile combination such as {\LARGE\wan{5}\wan{5}\wan{7}} that is made up with one protorun plus one floating tile ({\LARGE\wan{5}\wan{7}} + {\LARGE\wan{5}}).\footnote{Alternatively, we can think of these combinations as a pair plus one {\wan{5}\wan{5} + \wan{7}}.}
Depending on the type of protoruns, we can classify protorun plus one shapes into three types, as summarized in Table \ref{tbl:protoone}. 

\bigskip

{\begin{table}[h!]\centering\small \captionsetup{font=footnotesize}
\caption{Types of protorun plus one shapes} \label{tbl:protoone}
\begin{tabular}{l c c c}
\toprule
Name & Example & Wait & Acceptance\\
\midrule
side wait +1 & {\LARGE \wan{3}\wan{3}\wan{4}} & {\LARGE \wan{2}-\wan{5} \wan{3}} & 3 kinds--10 tiles\\ [\sep]
closed wait +1 & {\LARGE \tong{2}\tong{2}\tong{4}} & {\LARGE \tong{2} \tong{3}} & 2 kinds--6 tiles\\ [\sep]
edge wait +1 & {\LARGE \suo{8}\suo{8}\suo{9}} & {\LARGE \suo{7} \suo{8}} & 2 kinds--6 tiles\\ [\sep]
\bottomrule
\end{tabular}
\end{table}}

A protorun plus one can accept two additional tiles that an isolated protorun cannot. This is because these blocks can now be a candidate for a set as well as for a run. 

\bigskip
Breaking a protorun plus one can be inefficient. For example, if we break a closed wait plus one shape into an isolated pair (i.e., discard {\LARGE\tong{4}} from {\LARGE \tong{2}\tong{2}\tong{4}}), the tile acceptance decreases from 6 to 2; it can accept only {\LARGE\tong{2}} (1 kind--2 tiles). Similarly, if we break it into an isolated protorun (i.e., discard {\LARGE\tong{2}} from {\LARGE \tong{2}\tong{2}\tong{4}}), the tile acceptance decreases from 6 to 4; it can accept only {\LARGE\tong{3}} (1 kind--4 tiles). With this in mind, consider the following hand.

\begin{itembox}[r]{Protorun plus one}
\bp
\wan{1}\wan{1}\wan{5}\wan{7}\tong{5}\suo{1}\suo{2}\suo{3}\suo{4}\suo{7}\suo{8}\suo{9}\suo{9}~\wan{5}\\
\hfill\footnotesize{Draw~~~~~~~~~~~}
\ep
\vspace{-15pt}What would you discard? \vspace{-5pt}
\end{itembox}
\noindent
Discarding {\LARGE\wan{5}} or {\LARGE\wan{7}} to break the protorun plus one {\LARGE\wan{5}\wan{5}\wan{7}} is inefficient here. Discarding {\LARGE\wan{5}} decreases tile acceptance by two, and discarding {\LARGE\wan{7}} decreases tile acceptance by four. Moreover, discarding {\LARGE\wan{7}} leaves three pairs in this hand, which should be avoided. Discarding {\LARGE\suo{9}} is much more efficient. 

\bigskip

Sometimes we have to make a choice between multiple protorun plus one shapes, just like we did in examples in Section \ref{sec:ptat}. 
Consider the following hand. What would you discard?
\begin{itembox}[r]{Multiple protoruns plus one}
\bp
\wan{4}\wan{4}\wan{6}\tong{3}\tong{3}\tong{4}\suo{1}\suo{2}\suo{3}\suo{4}\suo{6}\suo{8}\bei\bei
\ep
\vspace{-10pt}What would you discard? \vspace{-5pt}
\end{itembox}
\noindent
There are two protorun plus one shapes in this hand: {\LARGE\wan{4}\wan{4}\wan{6}} and {\LARGE\tong{3}\tong{3}\tong{4}}. We have to break one of the two into either an isolated Pair or an isolated protorun, because the other parts of this hand are more or less self-sufficient. Which one should we choose?

\bigskip
When choosing between which protoruns plus one to break, priority should be given to the weaker one. Since the side-wait protorun {\LARGE\tong{3}\tong{4}} is much stronger than the closed-wait protorun {\LARGE\wan{4}\wan{6}}, we should prioritize the latter and maintain {\LARGE\wan{4}\wan{4}\wan{6}}. In other words, the side-wait protorun {\LARGE\tong{3}\tong{4}} is so strong that we do not need to provide a cover by maintaining the ``plus one'' tile, {\LARGE\tong{3}}. On the other hand, the closed-wait protorun {\LARGE\wan{4}\wan{6}} is weaker so we should cover it by keeping another {\LARGE\wan{4}} as a back-up. You should thus discard {\LARGE\tong{3}}. 

\subsection{Les formes complexes à quatre tuiles} 
Among several different kinds of four-tile complex shapes, we will focus on those that are made up of one group and one floating tile. There are three variants of this kind --- stretched single, bulging float, and skipping.

\subsubsection{Stretched single ({\jap nobetan}) shape} \label{sec:nobetan}
	\index{stretched single shape@stretched single ({\jap nobetan}) shape} 
	\index{nobetan@{\jap nobetan} (stretched single) shape}
A set of four consecutive tiles such as {\LARGE\tong{2}\tong{3}\tong{4}\tong{5}} is called a {\bf stretched single} ({\jap nobetan}) shape. Stretched single shapes are very useful both when a hand is ready and when a hand is 1-away or worse. 

\bigskip
When a stretched single shape is in a ready hand, that part forms the wait of the hand. For example, the following hand is ready, waiting for {\LARGE\wan{1} \wan{4}}.
\bigskip
\begin{itembox}[r]{Ready hand with a stretched single shape}
\bp
\wan{1}\wan{2}\wan{3}\wan{4}\tong{2}\tong{3}\tong{4}\suo{2}\suo{3}\suo{4}\zhong\zhong\zhong
\ep 
\end{itembox}
In a ready hand, the stretched single shape can be thought of as a candidate for the head ({\LARGE\wan{1}} or {\LARGE\wan{4}}) and a candidate for a run ({\LARGE\wan{2}\wan{3}\wan{4}} or {\LARGE\wan{1}\wan{2}\wan{3}}). For example, if we win this hand on {\LARGE\wan{1}}, then {\LARGE\wan{1}} becomes the head, and {\LARGE\wan{2}\wan{3}\wan{4}} becomes a run. On the other hand, if we win this hand on {\LARGE\wan{4}}, then {\LARGE\wan{4}} becomes the head, and {\LARGE\wan{1}\wan{2}\wan{3}} becomes a run. 

\bigskip

Another important role that a stretched single shape can play is to work as a candidate for two runs. When a hand is 1-away or worse, we can count on a stretched single shape to produce two runs. 
For example, consider a stretched single shape {\LARGE\tong{3}\tong{4}\tong{5}\tong{6}}. 
If we draw {\LARGE\tong{4}}, we will have a side-wait protorun {\LARGE\tong{3}\tong{4}} in addition to a complete run {\LARGE\tong{4}\tong{5}\tong{6}}. Similarly, if we draw {\LARGE\tong{5}}, we will have a side-wait protorun {\LARGE\tong{5}\tong{6}} in addition to a complete run {\LARGE\tong{3}\tong{4}\tong{5}}. Moreover, if we draw {\LARGE\tong{2}} or {\LARGE\tong{7}}, we will have a 3-way side-wait shape {\LARGE\tong{2}\tong{3}\tong{4}\tong{5}\tong{6}} (waiting for {\LARGE\tong{1}-\tong{4}-\tong{7}}) or {\LARGE\tong{3}\tong{4}\tong{5}\tong{6}\tong{7}} (waiting for {\LARGE\tong{2}-\tong{5}-\tong{8}}). 

\bigskip
There are six patterns of stretched single shapes, from 1234 through 6789. Table \ref{tbl:nobetan} summarizes the tiles each shape can accept to produce various waits. 

{\begin{table}[h!]\centering\small\captionsetup{font=footnotesize}
\caption{Types of stretched single shapes} \label{tbl:nobetan}
\begin{tabular}{l llll c}
\toprule
Shape & 3-way & 2-way & 1-way & Pair & Acceptance\\
\midrule
{\LARGE\wan{1}\wan{2}\wan{3}\wan{4}}
	&
	& {\LARGE \wan{3} \wan{5}}
	& {\LARGE \wan{2} \wan{6}}
	& {\LARGE \wan{1} \wan{4}}
	& 6 kinds--20 tiles\\ [\sep]
{\LARGE\wan{2}\wan{3}\wan{4}\wan{5}}
	& {\LARGE\wan{6}}
	& {\LARGE \wan{1} \wan{3} \wan{4}}
	& {\LARGE \wan{7}}
	& {\LARGE \wan{2} \wan{5}}
	& 7 kinds--24 tiles\\ [\sep]
{\LARGE\wan{3}\wan{4}\wan{5}\wan{6}}
	& {\LARGE\wan{2} \wan{7}}
	& {\LARGE \wan{4} \wan{5}}
	& {\LARGE \wan{1} \wan{8}}
	& {\LARGE \wan{3} \wan{6}}
	& 8 kinds--28 tiles\\ [\sep]
{\LARGE\wan{4}\wan{5}\wan{6}\wan{7}}
	& {\LARGE\wan{3} \wan{8}}
	& {\LARGE \wan{5} \wan{6}}
	& {\LARGE \wan{2} \wan{9}}
	& {\LARGE \wan{4} \wan{7}}
	& 8 kinds--28 tiles\\ [\sep]
{\LARGE\wan{5}\wan{6}\wan{7}\wan{8}}
	& {\LARGE\wan{4}}
	& {\LARGE \wan{6} \wan{7} \wan{9}}
	& {\LARGE \wan{3}}
	& {\LARGE \wan{5} \wan{8}}
	& 7 kinds--24 tiles\\ [\sep]
{\LARGE\wan{6}\wan{7}\wan{8}\wan{9}}
	&
	& {\LARGE \wan{5} \wan{7}}
	& {\LARGE \wan{4} \wan{8}}
	& {\LARGE \wan{6} \wan{9}}
	& 6 kinds--20 tiles\\ [\sep]
\bottomrule
\end{tabular}
\end{table}}

As we can see, the middle two ones --- 3456 and 4567 --- are the most versatile. They can accept two different tiles to produce a 3-way wait (27 or 38), two different tiles to produce a 2-way side wait (45 or 56), and two different tiles to produce a 1-way wait (18 or 29 to produce a closed wait). The 3456 and 4567 shapes are the most valuable of all four-tile shapes, and we should not lightly break such shapes when a hand is far away from ready. With this in mind, consider the following 2-away hand. 

\vfill
\bigskip
\begin{itembox}[r]{2-away hand with a stretched single shape}
\bp
\wan{3}\wan{4}\wan{5}\wan{6}\tong{1}\tong{1}\tong{3}\tong{5}\suo{1}\suo{2}\suo{6}\suo{6}\suo{7}\suo{9}
\ep
\vspace{-10pt}What would you discard? \vspace{-5pt}
\end{itembox}

\bigskip 
\noindent
It is true that discarding {\LARGE\wan{3}} or {\LARGE\wan{6}} would lead to the greatest tile acceptance (7 kinds--24 tiles) temporarily. However, doing so is too myopic. If we do that, all the remaining protoruns will be closed-wait or edge-wait ones. We should rather discard {\LARGE\suo{9}} to keep the 3456 shape, which we can expect to produce two side-wait protoruns later. The resulting tile acceptance (6 kinds--20 tiles) is not much smaller, either. 

\bigskip

\color{MyRed}
\begin{itembox}[c]{Four-tile complex shapes 1: {\jap nobetan}}
\normalcolor
Try to keep a stretched single shape if a hand has one. In particular, 3456 and 4567 should be kept until the hand becomes ready or 1-away from ready. 
\end{itembox}
\normalcolor

\bigskip


\subsubsection{Bulging float ({\jap nakabukure}) shape}
	\index{bulging@bulging float ({\jap nakabukure})} 
	\index{nakabukure@{\jap nakabukure} (bulging float)}

When we have a floating tile in the middle of a run (e.g., {\LARGE\wan{3}\wan{4}\wan{4}\wan{5}}), we have a {\bf bulging float} ({\jap nakabukure}) shape. 
Bulging float shapes are quite good at producing side-wait protoruns. Any bulging float shapes from 2334 through 6778 can accept four kinds of tiles to produce a side-wait protorun and a complete run. Take {\LARGE\wan{3}\wan{4}\wan{4}\wan{5}}, for example. It can produce a side-wait protorun and a complete run if we draw any of {\LARGE\wan{2}\wan{3}\wan{5}\wan{6}}. 
With this in mind, consider the following 2-away hand.
\bigskip
\begin{itembox}[r]{Hand with a bulging float shape}
\bp
\wan{3}\wan{4}\wan{4}\wan{5}\tong{4}\tong{6}\tong{8}\tong{8}\suo{1}\suo{3}\suo{5}\suo{5}\suo{6}\suo{8}
\ep
\vspace{-10pt}What would you discard? \vspace{-5pt}
\end{itembox}
Discarding {\LARGE\wan{4}} to break the bulging float shape is not ideal. Although doing so increases tile acceptance temporarily, the hand will be filled with closed-wait protoruns. Alternatively, you should discard {\LARGE\suo{8}} to maintain the bulging float shape. 

\bigskip
That being said, when this shape remains as is when a hand is ready, it does not make for a good wait. For example, consider the following ready hand. 
\begin{itembox}[r]{Ready hand with a bulging float shape}
\bp
\wan{3}\wan{4}\wan{4}\wan{5}\tong{2}\tong{3}\suo{2}\suo{3}\suo{4}\bei\zhong\zhong\zhong~\tong{4}\\
\hfill\footnotesize{Draw~~~~~~~~~~~}
\ep 
\vspace{-15pt}What would you discard? \vspace{-5pt}
\end{itembox}
\noindent
Discarding {\LARGE\bei} to keep the bulging float shape {\LARGE\wan{3}\wan{4}\wan{4}\wan{5}} makes the wait of this hand pretty bad. It is waiting for {\LARGE\wan{4}}, but we are already using two of it in the hand, leaving only two winning tiles. We should rather discard {\LARGE\wan{4}} to wait for {\LARGE\bei}. 

\vfill
\color{MyRed}
\begin{itembox}[c]{Four-tile complex shapes 2: {\jap nakabukure}}
\normalcolor
Try to keep a bulging float shape until a hand becomes 1-away.
\end{itembox}\normalcolor

\subsubsection{Skipping shape}
\index{skipping shape}
When we have a floating tile two tiles away from a run, we have a {\bf skipping shape}. For example, in a shape {\LARGE\wan{3}\wan{5}\wan{6}\wan{7}}, {\LARGE\wan{3}} is floating next next to a run {\LARGE\wan{5}\wan{6}\wan{7}}. {\LARGE\wan{3}} in a skipping shape is more valuable than isolated {\LARGE\wan{3}}, because it increases the kinds of tiles the hand can accept to produce a protorun or a 3-way side-wait shape. Table \ref{tbl:skipping} summarizes all the skipping shapes and the tiles each shape can accept. 

{\begin{table}[h!]\centering\small\captionsetup{font=footnotesize}
\caption{Types of skipping shapes} \label{tbl:skipping}
\begin{tabular}{l llll c}
\toprule
Shape & 3-way & 2-way & 1-way & Pair & Acceptance\\
\midrule
{\LARGE\wan{1}\wan{3}\wan{4}\wan{5}}
	&
	& {\LARGE \wan{2}}
	& {\LARGE \wan{3} \wan{6}}
	& {\LARGE \wan{1}}
	& 4 kinds--14 tiles\\ [\sep]
{\LARGE\wan{2}\wan{4}\wan{5}\wan{6}}
	& {\LARGE\wan{3}}
	& 
	& {\LARGE \wan{1} \wan{4} \wan{7}}
	& {\LARGE \wan{2}}
	& 5 kinds--18 tiles\\ [\sep]
{\LARGE\wan{3}\wan{5}\wan{6}\wan{7}}
	& {\LARGE\wan{4}}
	& {\LARGE\wan{2}}
	& {\LARGE \wan{1} \wan{5} \wan{8}}
	& {\LARGE \wan{3}}
	& 6 kinds--22 tiles\\ [\sep]
{\LARGE\wan{4}\wan{6}\wan{7}\wan{8}}
	& {\LARGE\wan{5}}
	& {\LARGE\wan{3}}
	& {\LARGE \wan{2} \wan{6} \wan{9}}
	& {\LARGE \wan{4}}
	& 6 kinds--22 tiles\\ [\sep]
{\LARGE\wan{5}\wan{7}\wan{8}\wan{9}}
	& 
	& {\LARGE\wan{4} \wan{6}}
	& {\LARGE\wan{3} \wan{7}}
	& {\LARGE \wan{5}}
	& 5 kinds--18 tiles\\ [\sep]
{\LARGE\wan{1}\wan{2}\wan{3}\wan{5}}
	& 
	& {\LARGE \wan{4} \wan{6}}
	& {\LARGE \wan{3} \wan{7}}
	& {\LARGE \wan{5}}
	& 5 kinds--18 tiles\\ [\sep]
{\LARGE\wan{2}\wan{3}\wan{4}\wan{6}}
	& {\LARGE\wan{5}}
	& {\LARGE\wan{7}}
	& {\LARGE \wan{1} \wan{8}}
	& {\LARGE \wan{6}}
	& 6 kinds--22 tiles\\ [\sep]
{\LARGE\wan{3}\wan{4}\wan{5}\wan{7}}
	& {\LARGE\wan{6}}
	& {\LARGE\wan{8}}
	& {\LARGE \wan{2} \wan{5} \wan{9}}
	& {\LARGE \wan{7}}
	& 6 kinds--22 tiles\\ [\sep]
{\LARGE\wan{4}\wan{5}\wan{6}\wan{8}}
	& {\LARGE\wan{7}}
	& 
	& {\LARGE \wan{3} \wan{6} \wan{9}}
	& {\LARGE \wan{8}}
	& 5 kinds--18 tiles\\ [\sep]
{\LARGE\wan{5}\wan{6}\wan{7}\wan{9}}
	&
	& {\LARGE \wan{8}}
	& {\LARGE \wan{4} \wan{7}}
	& {\LARGE \wan{9}}
	& 4 kinds--14 tiles\\ [\sep]
\bottomrule
\end{tabular}
\end{table}}

\bigskip
Bearing in mind that {\LARGE\wan{3}} of {\LARGE\wan{3}\wan{5}\wan{6}\wan{7}} is more valuable than isolated {\LARGE\wan{3}}, consider the following hand. 
\begin{itembox}[r]{Hand with a skipping shape}
\bp
\wan{3}\wan{7}\wan{8}\tong{5}\tong{6}\tong{7}\suo{2}\suo{4}\suo{6}\suo{7}\zhong\zhong\zhong~\tong{3}\\
\hfill\footnotesize{Draw~~~~~~~~~~~}
\ep
\vspace{-15pt}What would you discard? \vspace{-5pt}
\end{itembox}
\noindent
We should keep {\LARGE\tong{3}} and discard {\LARGE\wan{3}} instead. This is because {\LARGE\tong{3}} is a part of a skipping shape {\LARGE\tong{3}\tong{5}\tong{6}\tong{7}}, but {\LARGE\wan{3}} is an isolated floating tile. 

\bigskip
As we can see in Table \ref{tbl:skipping}, skipping shapes with a terminal tile (1345 and 5679) are also valuable. The 1 of 1345 and the 9 of 5679 can accept more tiles than an isolated 2 or 8 (let alone than an isolated 1 or 9). 

\newpage
\section{Waits} \label{sec:waits}
	\index{waits}
	\index{waits!side wait} \index{waits!closed wait} 
	\index{waits!dual pon wait} \index{waits!edge wait} \index{waits!single wait} 

There are five basic wait patterns, as summarized in Table \ref{tbl:waits}. More complicated wait patterns can emerge when some of these five basic patterns are combined. 

{\begin{table}[h!]\centering\small\captionsetup{font=footnotesize}
\caption{Five basic waits} \label{tbl:waits}
\begin{tabular}{l l c c c}
\toprule
Name & Japanese & Example & Wait & Acceptance\\
\midrule
side wait & {\jap ryanmen} & {\LARGE \wan{3}\wan{4}} & {\LARGE \wan{2}-\wan{5}} & 2 kinds--8 tiles\\ [\sep]
dual {\jap pon} wait & {\jap shanpon} & {\LARGE \suo{3}\suo{3}\tong{5}\tong{5}}& {\LARGE \suo{3} \tong{5}} & 2 kinds--4 tiles\\ [\sep]
closed wait & {\jap kanchan} & {\LARGE \suo{6}\suo{8}} & {\LARGE \suo{7}} & 1 kind--4 tiles\\ [\sep]
edge wait & {\jap penchan} & {\LARGE \tong{1}\tong{2}} & {\LARGE \tong{3}} & 1 kind--4 tiles\\ [\sep]
single wait & {\jap tanki} & {\LARGE \wan{2}} & {\LARGE \wan{2}} & 1 kind--3 tiles\\ [\sep]
\bottomrule
\end{tabular}
\end{table}}

\bigskip
As we can see in the table, side wait is the strongest of all the basic waits in terms of the kinds and the number of tiles to win on. 
Single wait appears to be much worse than others, but single-wait hands tend to have many possibilities of improving the wait and/or scores further. Moreover, single wait of an honor tile has a relatively high chance of winning it by {\jap ron}.

\subsubsection{Stretched single wait and semi side wait}
	\index{waits!stretched single wait} 
	\index{waits!semi side wait} 
Table \ref{tbl:waits2} summarizes two wait patterns, each of which can be thought of as a combination of some basic wait patterns. 
As I mentioned before, a stretched single shape in a ready hand forms a 2-way single wait. It is a decent wait pattern, as the number of tiles to win on (2 kinds--6 tiles) is twice as big compared with a regular single wait. 

{\begin{table}[h!]\centering\small\captionsetup{font=footnotesize}
\caption{Stretched single wait and semi side wait} \label{tbl:waits2}
\begin{tabular}{l c c c}
\toprule
Name & Example & Waits & Acceptance\\
\midrule
stretched single wait & {\LARGE \wan{2}\wan{3}\wan{4}\wan{5}} & {\LARGE \wan{2} \wan{5}} & 2 kinds--6 tiles\\ [\sep]
semi side wait & {\LARGE \suo{3}\suo{3}\suo{4}\suo{5}}& {\LARGE \suo{3}-\suo{6}} & 2 kinds--6 tiles\\ [\sep]
\bottomrule
\end{tabular}
\end{table}}

\bigskip
However, stretched single wait should not be confused with side wait for a few reasons. First, the number of tiles a 2-way stretched-single-wait hand can win on is at most 6, whereas it is 8 for a 2-way side-wait hand. The difference between 6 and 8 is non-trivial. Second, stretched single wait is still a variant of single wait, which means two things. On the one hand, we cannot claim {\jap pinfu} when the wait is stretched-single wait. For example, the following hand has no {\jap yaku} and hence we cannot win it by {\jap ron} without calling riichi.  
\bp
\wan{4}\wan{5}\wan{6}\tong{1}\tong{2}\tong{3}\tong{4}\suo{1}\suo{2}\suo{3}\suo{4}\suo{5}\suo{6}
\ep \index{fu@{\jap fu} (minipoint)} \index{minipoint ({\jap fu})}
On the other hand, we get 2 minipoints ({\jap fu}) with a stretched single wait. 
For example, if we win the following hand by drawing {\LARGE\tong{1}}, we get 40 minipoints (20 base minipoints + 8 for a concealed set of honor tiles + 2 for self-draw + 2 for single wait = 32, rounded up to 40).\footnote{We will discuss methods of scoring and minipoints calculations extensively in Chapter \ref{ch:scores}.}
\bp
\wan{4}\wan{5}\wan{6}\tong{1}\tong{2}\tong{3}\tong{4}\suo{1}\suo{2}\suo{3}\zhong\zhong\zhong
\ep

When we have a side-wait protorun right next to a pair (e.g., 1123, 2234, 7899, etc.), we call it semi side wait. We distinguish this from regular side wait for two reasons. First, the number of tiles to win on is smaller (6 rather than 8) because we are already using 2 of the 8 winning tiles in our hand. Second, we can treat this wait pattern either as single wait or as side wait, depending on which interpretation gives us a greater score. For example, consider the following hand. 
\bp
\wan{4}\wan{5}\wan{6}\tong{1}\tong{1}\tong{2}\tong{3}\suo{1}\suo{2}\suo{3}\suo{4}\suo{5}\suo{6}
\ep
We will treat the wait in this hand as side wait because that will give us {\jap pinfu}.
However, consider the following hand that has the exact same wait pattern: {\LARGE\tong{1}\tong{1}\tong{2}\tong{3}}. 
\bp
\wan{4}\wan{5}\wan{6}\tong{1}\tong{1}\tong{2}\tong{3}\suo{1}\suo{2}\suo{3}\zhong\zhong\zhong
\ep
If we win this hand by drawing {\LARGE\tong{1}}, we will treat the wait as single wait: {\LARGE\tong{1}} + {\LARGE\tong{1}\tong{2}\tong{3}}, which will give us 40 minipoints. If we treated the wait as side wait: {\LARGE\tong{1}\tong{1}} + {\LARGE\tong{2}\tong{3}}, we would get only 30 minipoints.
Of course, if we win this hand on {\LARGE\tong{4}}, we cannot think of the wait as side wait (because it is not). Similarly, if we win it by {\jap ron}, it does not make a difference if it is side wait or single wait (either way we get 40 minipoints). 

\newpage

\subsubsection{3-way side wait}
When a side-wait protorun is combined with an adjacent run, we get a regular 3-way side-wait pattern. There are only three of this kind, summarized in Table \ref{tbl:waits3}.

{\begin{table}[h!]\centering\captionsetup{font=footnotesize}\small
\caption{Regular 3-way side wait} \label{tbl:waits3}
\begin{tabular}{l c c}
\toprule
Example & Wait & Acceptance\\
\midrule
{\LARGE \wan{2}\wan{3}\wan{4}\wan{5}\wan{6}} & {\LARGE \wan{1}-\wan{4}-\wan{7}} & 3 kinds--11 tiles\\ [\sep]
{\LARGE \tong{3}\tong{4}\tong{5}\tong{6}\tong{7}} & {\LARGE \tong{2}-\tong{5}-\tong{8}} & 3 kinds--11 tiles\\ [\sep]
{\LARGE \suo{4}\suo{5}\suo{6}\suo{7}\suo{8}} & {\LARGE \suo{3}-\suo{6}-\suo{9}} & 3 kinds--11 tiles\\ [\sep]
\bottomrule
\end{tabular}
\end{table}}

\bigskip
When we have a stretched single shape or semi side-wait shape combined with an adjacent run, we also get a 3-way wait pattern. Table \ref{tbl:waits4} summarizes some examples. 

\bigskip
{\begin{table}[h!]\centering\captionsetup{font=footnotesize}\small
\caption{Some irregular 3-way waits} \label{tbl:waits4}
\begin{tabular}{l c c}
\toprule
Example & Wait & Acceptance\\
\midrule
{\LARGE \wan{1}\wan{2}\wan{3}\wan{4}\wan{5}\wan{6}\wan{7}} & {\LARGE \wan{1} \wan{4} \wan{7}} & 3 kinds--9 tiles\\ [\sep]
{\LARGE \tong{2}\tong{3}\tong{4}\tong{5}\tong{5}\tong{6}\tong{7}} & {\LARGE \tong{2} \tong{5}-\tong{8}} & 3 kinds--9 tiles\\ [\sep]
{\LARGE \suo{4}\suo{5}\suo{6}\suo{7}\suo{8}\suo{9}\suo{9}} & {\LARGE \suo{3}-\suo{6}-\suo{9}} & 3 kinds--9 tiles\\ [\sep]
\bottomrule
\end{tabular}
\end{table}}

\bigskip
Notice that the number of tiles to win on in each pattern is smaller than those for the regular 3-way side waits, although the kinds of tiles to win on are the same (either 1-4-7, 2-5-8, or 3-6-9). This is because we are already using some of the winning tiles within the hand. 

\bigskip
Notice also that not all the wait patterns qualify as side wait, so claiming {\jap pinfu} is not always possible (similarly, claiming single wait is not always possible). For example, the first pattern in Table \ref{tbl:waits4} is essentially a 3-way stretched single shape; none of the waits embedded in this shape qualifies as side wait. In the second pattern, if we win on {\LARGE\tong{2}}, the wait must be interpreted as single wait; if we win on {\LARGE\tong{8}}, the wait must be interpreted as side wait; and if we win on {\LARGE\tong{5}}, we adopt whichever interpretation that generates the higher score. In the third pattern, winning on {\LARGE\suo{9}} allows us to claim single wait if doing so gives us a higher score. 

\vfill
\subsubsection{Complex waits}
When a set is combined with a floating tile nearby, we get some complex wait patterns with multiple waits. Table \ref{tbl:waits5} summarizes a few examples of irregular waits that involve a set and a floating tile. 

{\begin{table}[h!]\centering \small\captionsetup{font=footnotesize}
\caption{Some irregular waits (set and a floating tile)} \label{tbl:waits5}
\begin{tabular}{l c c c}
\toprule
Example & Combination & Wait & Acceptance\\
\midrule
{\LARGE \wan{1}\wan{2}\wan{2}\wan{2}} & single and edge & {\LARGE \wan{1} \wan{3}} & 2 kinds--7 tiles\\ [\sep]
{\LARGE \tong{1}\tong{3}\tong{3}\tong{3}} & single and closed & {\LARGE \tong{1} \tong{2}} & 2 kinds--7 tiles\\ [\sep]
{\LARGE \suo{2}\suo{3}\suo{3}\suo{3}} & single and side & {\LARGE \suo{2} \suo{1}-\suo{4}} & 3 kinds--11 tiles\\ [\sep]
\bottomrule
\end{tabular}
\end{table}}


\bigskip
When a set is combined with a protorun, pair, or a four-tile shape, we get even more complicated waits. Table \ref{tbl:waits6} summarizes only a few representative examples. 

{\begin{table}[t!]\centering \small \captionsetup{font=footnotesize}
\caption{Some irregular waits (set and a protorun, pair, or a four-tile shape)} \begin{tabular}{l c c}
\toprule
Example & Wait & Acceptance\\
\midrule
{\LARGE \wan{1}\wan{1}\wan{2}\wan{2}\wan{2}\wan{3}\wan{3}} & {\LARGE \wan{1} \wan{2} \wan{3}} & 3 kinds--5 tiles\\ [\sep]
{\LARGE \tong{1}\tong{1}\tong{2}\tong{2}\tong{3}\tong{3}\tong{3}} & {\LARGE \tong{1} \tong{2} \tong{3}} & 3 kinds--5 tiles\\ [\sep]
{\LARGE \suo{1}\suo{1}\suo{2}\suo{2}\suo{3}\suo{3}\suo{4}\suo{4}\bei\bei} & {\LARGE \suo{1} \suo{4} \bei} & 3 kinds--6 tiles\\ [\sep]
{\LARGE \wan{5}\wan{5}\wan{5}\wan{6}\wan{7}\nan\nan} & {\LARGE \wan{5}-\wan{8} \nan} & 3 kinds--7 tiles\\ [\sep]
{\LARGE \tong{6}\tong{7}\tong{8}\tong{8}\tong{9}\tong{9}\tong{9}} & {\LARGE \tong{5}-\tong{8} \tong{7}} & 3 kinds--9 tiles\\ [\sep]
{\LARGE \suo{2}\suo{2}\suo{2}\suo{3}\suo{4}\suo{4}\suo{5}} & {\LARGE \suo{3}-\suo{6} \suo{4}} & 3 kinds--9 tiles\\ [\sep]
{\LARGE \wan{3}\wan{3}\wan{3}\wan{5}\wan{6}\wan{7}\wan{8}} & {\LARGE \wan{4} \wan{5} \wan{8}} & 3 kinds--10 tiles\\ [\sep]
{\LARGE \tong{1}\tong{1}\tong{1}\tong{3}\tong{5}\tong{5}\tong{5}} & {\LARGE \tong{2} \tong{3} \tong{4}} & 3 kinds--11 tiles\\ [\sep]
\bottomrule
\end{tabular}
\label{tbl:waits6}
\end{table}}


\clearpage

\section{Glossary}

\begin{description}
\item[Simple tiles ({\jap tanyao hai})] are tiles between 2 and 8.
\item[Terminal tiles] are 1 and 9.
\item[Honor tiles ({\jap jihai})] are non-number tiles (dragon tiles and wind tiles).
\item[Value tiles ({\jap fanpai} / {\jap yakuhai})] include dragon tiles, seat wind tiles, and prevailing wind tiles. We get one {\jap han} for a set of value tiles. 
\item[Valueless wind tiles ({\jap otakaze hai})] are wind tiles that are neither a prevailing wind tile nor a seat wind tile.
\item[Run (chow / sequence; {\jap shuntsu})] is a set of three consecutive number tiles.
\item[Set (pung / triplet; {\jap kotsu})] is a set of three identical tiles. 
\item[Quad (kong; {\jap kantsu})] is a set of four identical tiles. 
\item[Protorun ({\jap taatsu})] is a set of two tiles in the same suit that can become a run when one more tile is added. 
\item[Pair ({\jap toitsu})] is a set of two identical tiles. 
\item[Ready ({\jap tenpai})] is when a hand is ready to win. 
\item[1-away ({\jap 1-shanten})] is when a hand can be ready with one more tile. 
\item[Perfect 1-away] is when a 1-away hand has two side-wait protoruns and two pairs.
\item[Tile acceptance ({\jap ukeire})] refers to the kinds and the number of tiles a hand can accept.
\item[Stretched single ({\jap nobetan}) shape] is a set of four consecutive number tiles. 
\item[Bulging float ({\jap nakabukure}) shape] is a four-tile shape that is made up with a run and one floating tile in the middle of the run.
\item[Skipping shape] is a four-tile shape made up with a run and one floating tile located at two tiles away from the run.
\end{description}
