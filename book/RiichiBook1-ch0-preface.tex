%~~~~~~~~~~~~~~~~~~~~~~~~~~~~~~~~~~~~~~~~~~~~~~~~~
% Riichi Book 1, Préface
%~~~~~~~~~~~~~~~~~~~~~~~~~~~~~~~~~~~~~~~~~~~~~~~~~
\chapter{Préface}
\thispagestyle{empty}

Lorsqu'en 2013 j'ai emménagé en Angleterre, j'ai été agréablement surpris d'apprendre que le riichi mah-jong (mah-jong japonais moderne) était assez populaire en Europe. Au cours des deux dernières années, j'ai eu le plaisir de jouer au riichi à Londres, à Guildford, à Kent, à Oxford, à Aix-la-Chapelle, à Copenhague,\\
à Prague, et à Vienne, avec des joueurs d'Austriche, de Chine, de République tchèque, du Danemark, d'Estonie, de Finlande, de France, d'Allemagne, d'Italie, du Japon, des Pays-Bas, de Pologne, de Russie, de Slovaquie, de Suède, du Royaume-Uni, et des États-Unis.

\bigskip
Les joueurs européens ont su organiser des tournois ouverts à tous. Ces tournois --- qui ont lieu une fois par mois quelque part en Europe --- sont organisés par les joueurs de chaque pays sous la supervision de l'European Mahjong Association (EMA).\footnote{\url{http://mahjong-europe.org/}} \index{european@EMA}
Fondée en 2005, l'EMA a fait un travail fantastique en instaurant des règles communes%
\footnote{Les règles officielles de l'EMA sont disponibles en ligne sur \url{http://mahjong-europe.org/portal/images/docs/Riichi-rules-2016-EN.pdf} (dernière révision en 2016). Au moment de la rédaction de ce livre, l'EMA est en train de réviser les règles.
Les explications des règles de l'EMA dans ce livre sont basées sur les règles révisées. Les nouvelles règles entreront en vigueur en avril 2016.
}
en instaurant un système de classement des joueurs et promouvant le mah-jong à travers l'Europe. 

\bigskip
Même si j'ai rencontré de bon joueur en Europe, je me suis rendu compte que beaucoup de joueurs ici ne connaissent pas les principes de base des stratégies de mah-jong compétitifs. Evidemment que de jouer de manière compétitive n'est pas la seule façon d'apprécier le jeu.
Je ne prétends pas non plus connaître la formule magique pour gagner, cela n'existe pas. Néanmoins, il existe un ensemble de principes de base qui méritent d'être appris par tous les joueurs. Je pense que le niveau global des joueurs européens pourrait être grandement augmenté si ces principes étaient largement partagés. Malheureusement, les ressources pédagogiques pour le public non japanais sont assez limitées.\footnote{Il existe déjà quelques livres en anglais pour les débutants. Il existe également d'excellents articles de blog sur les détails techniques des stratégies de mah-jong. Cependant, il semble qu'il y ait un énorme fossé entre ces deux types de ressources. Les livres d'introduction ne couvrent pas les stratégies de manière exhaustive, tandis que les articles de blog ont tendance à être trop avancés, même pour les joueurs de niveau intermédiaire.}

\bigskip
J'ai donc décidé d'écrire un livre sur les stratégies de riichi pour les joueurs européens, principalement à l'intention des débutants et des joueurs intermédiaires. J'ai donc fini par diviser le livre en deux volumes ; le livre I est destiné aux débutants et aux joueurs intermédiaires  (rang {\jap Tenhou} de 四段 ou moins), tandis que le livre II est destiné aux joueurs plus avancés. Les deux livres ne sont pas destinés aux novices complets qui ne savent pas comment jouer au riichi.\footnoteSi vous voulez apprendre à jouer au riichi, je vous recommande Barr (2009).} Le lecteur cible est toute personne ayant déjà joué au riichi et souhaitant améliorer ses compétences. 
\index{Barr@Barr, Jenn}

\bigskip
J'ai trois objectifs principaux en préparant ces livres. Premièrement, j'introduirai un ensemble de termes anglais relatifs au riichi mahjong. 
"Au commencement était la Parole", nous disent les Ecritures. Connaître les noms des combinaisons de tuiles, des situations et des stratégies particulières nous permettra d'y être attentifs et de pouvoir en parler avec nos compagnons de jeu après la partie. 

\bigskip
Mon deuxième objectif est de présenter les principes de l'efficacité des tuiles. 
Le Livre I et le Livre II traitent tous deux de l'efficacité des tuiles, mais à des niveaux différents. Le livre I propose une introduction à l'efficacité des tuiles, couvrant uniquement les mécanismes de base. Je prévois d'aborder des sujets plus avancés dans le livre II. 
Mon troisième objectif est d'introduire un ensemble de stratégies simples concernant des jugements critiques tels que le fait de se déclarer {\jap riichi} ou pas, le fait d'attaquer ou de se coucher, et le fait d'ouvrir ou non.

\bigskip
Une grande partie des sujets abordés dans les livres m'ont été présentés à travers les écrits d'un joueur de mahjong et auteur de manga japonais de renom, Masa\-yuki Kata\-yama. Mr.~Kata\-yama est un joueur de riichi accompli et sans doute le meilleur auteur de mangas de mahjong au monde. Certaines des stratégies présentées ici sont ouvertement tirées du chef-d'œuvre de Mr. Kata\-yama, le manga $Uta\-hime$ $Obaka\-miiko$ (『打姫\-オバカ\-ミーコ』). 
Je vous encourage vivement à le lire si vous lisez le japonais, même si je me rends compte que vous ne liriez pas mon livre si vous compreniez le japonais.
\index{Katayama@Katayama, Masayuki}

\bigskip
Makoto Fukuchi est un autre auteur japonais dont l'œuvre a influencé la rédaction du Livre I. Mr.~Fukuchi est également un joueur de riichi distingué et l'auteur le plus vendu de livres sur la stratégie du mahjong. Une partie de l'exposé de la méthode des cinq blocs dans le chapitre \ref{ch:five}~est basé sur l'explication détaillée des livres de Mr.~Fukuchi. \index{Fukuchi@Fukuchi, Makoto}

\bigskip
Je suis également redevable à de nombreux amis que j'ai rencontrés en jouant au mah-jong en Europe. Philipp Martin a lu une première version du livre et m'a fait part de ses commentaires et de ses encouragements. Je suis également reconnaissant à Gemma Sakamoto, qui a organisé une rencontre mensuelle de mah-jong à Londres. 
Enfin, je remercie Ian Fraser, l'un des fondateurs de l'association britannique de mah-jong. 
Sans les efforts de Ian et de son équipe, je n'aurais pas pu rencontrer autant de joueurs au Royaume-Uni et en Europe.

\bigskip
La photo de couverture (\copyright~Katar\'{i}na M\'{o}zov\'{a}) vient du Bratislava Riichi Open Tournament de 2015. Je remercie Katar\'{i}na et l'association slovaque de mah-jong (especially Matej Laba\v{s})de m'avoir donné leur permission de l'utiliser.

\bigskip
Après avoir rendu le livre accessible au public en janvier 2016, de nombreuses personnes m'ont fait part de leurs commentaires sur divers aspects de l'ouvrage. Sur la base de leurs commentaires, j'ai corrigé certaines incohérences terminologiques et fautes de frappe. Je remercie en particulier David Clarke, Aaron Ebejer, Nicolas Giaconia, Grant Mahoney, Ting et Chris Rowe pour leurs précieuses contributions.

\vfill

\hfill Daina Chiba\\
\hfill Londres, RU\\
\hfill 10 Janvier, 2016\\
\hfill (mis à jour le 10 avril, 2017)

\section*{Plan du livre}

Pour améliorer vos compétences, vous devez non seulement apprendre les théories, mais aussi mettre en pratique ce que vous avez appris en jouant de nombreuses parties, de préférence avec des joueurs plus forts que vous. Avant l'avènement des plateformes de mah-jong en ligne, il n'était toutefois pas très facile de le faire si l'on ne vivait pas au Japon. 

\bigskip
Grâce au développement récent des plateformes de mahjong en ligne, il est désormais possible de jouer des centaines ou des milliers de parties avec des adversaires compétents tout en vivant en dehors du Japon. Sur ces sites web, vous pouvez facilement trouver d'autres joueurs avec qui jouer 24 heures sur 24 et 7 jours sur 7. La plupart des plates-formes conservent l'historique de toutes les parties jouées par les joueurs, et une fonction de relecture vous permet de vous remémorer vos parties passées. Vous pouvez également consulter les statistiques des joueurs, qui vous donnent des indications importantes sur les axes de progression que vous devez travailler.  

\bigskip
Je vous recommande donc de vous entraîner au mahjong en jouant en ligne pendant que vous étudiez les principes de la stratégie avec ce livre. Vous n'avez pas besoin d'attendre d'avoir fini de lire le livre avant de commencer à jouer. Jouez d'abord quelques pqrties, puis revenez au livre et étudiez les chapitres pertinents. 

\bigskip

Le livre est divisé en quatre partie. Partie \ref{part:online} propose une introduction à une plateforme de mah-jong en ligne appelée {\jap Tenhou} (天鳳). Le site est en japonais, mais je vous guiderai à travers le processus d'inscription au site et vous montrerai comment jouer des parties au chapitre \ref{ch:Tenhou}. Il existe déjà d'excellentes ressources en ligne qui expliquent le fonctionnement de {\jap Tenhou}, notamment:
\bi \itemsep-.5em
\i Arcturus's Tenhou Documentation
\vspace{-10pt} \index{Arcturus}
	\bi
	\i[] \url{http://arcturus.su/tenhou/}
	\ei
	
\i Guide complet du Mah-jong en ligne pour les débutants (Osamuko)
\vspace{-10pt} \index{Osamuko}
	\bi
	\i[] \url{http://goo.gl/F5sJvO}
	\ei
\i Jouer en ligne: Tenhou (Reach Mahjong of New York)
	\bi \i[] \url{http://goo.gl/Oc1eNe} \ei
\ei
Si vous avez déjà lu l'un ou l'autre de ces trois ouvrages, vous pouvez sauter le chapitre \ref{ch:Tenhou} de ce livre, car il ne contient pas beaucoup d'informations nouvelles pour vous. Le chapitre \ref{ch:Tenhou2} explique certaines fonctions avancées de {\jap Tenhou}, que vous pouvez également ignorer si vous lisez ce livre pour la première fois.

\bigskip
Les parties \ref{part:tile} et \ref{part:strat} sont la "moelle" du livre. 
La partie \ref{part:tile} couvre les théories de base de l'efficacité des tuiles qui vous permettent de maximiser la vitesse de construction et/ou la valeur de votre main. Après avoir introduit la terminologie de base dans le chapitre \ref{ch:basic}, je traite de la méthode des cinq blocs dans le chapitre \ref{ch:five} et je donne quelques conseils sur la façon de réaliser plusieurs {\jap yaku} dans le chapitre \ref{ch:yaku}. 
La partie \ref{part:strat} couvre les principes de la stratégie, y compris les méthodes de calcul des scores (chapitre \ref{ch:scores}), l'appréciation du riichi (chapitre \ref{ch:riichi}), l'appréciation de la défense (chapitre \ref{ch:défense}), l'appréciation de l'ouverture de sa main (chapitre \ref{ch:call}), et ce que l'on appelle les "grandes stratégies" pour gagner une partie (chapitre \ref{ch:grand}). 
Enfin, les annexes comprennent un chapitre sur les bonnes manière du jeu hors ligne (chapitre \ref{ch:manners}) et un autre chapitre sur les lectures complémentaires (chapitre \ref{ch:read}). 

\bigskip
Les chiffres et les lettres affichés {\color{MyBlue} dans cette couleur} ainsi que chaque entrée dans la Table des matières ci-dessous sont liés par un hyperlien ; en cliquant sur l'un d'entre eux, vous accéderez à la page correspondante. 
%I omit page numbers to save space, but each page is given an implicit page number that a PDF reader (such as the Kindle app) would recognize. Each entry in the Index section at the end of the book also refers to such implicit page numbers. 

